In this chapter, we describe the logical framework used in
Chapters~\ref{chap:memo15}
and~\ref{chap:emsoft16}. Chapter~\ref{chap:exp-smpl} uses a different
framework, so the logical background is presented within that chapter. All
of our work is formalized inside the Coq proof assistant, but for
exposition purposes, we focus on the mathematical concepts rather than
concrete Coq syntax.

\section{Linear Temporal Logic}
In Chapters~\ref{chap:memo15} and~\ref{chap:emsoft16}, we encode
sampled-data systems and their properties within discrete-time linear
temporal logic (LTL).  An LTL formula specifies the possible traces of a
system.  In our model, a trace is an infinite sequence of states
representing observations of a system at discrete points in time.  A state
is a mapping from variables to real numbers.  Inspired by Lamport's
Temporal Logic of Actions (TLA)~\cite{lamport1994temporal}, formulas in our
logic are classified into \emph{state formulas} (predicates over a single
state), \emph{action formulas} (state relations specifying system
transitions), and \emph{trace formulas} (predicates over traces).  In
action formulas, the values of variables in the current state are notated
using bold script, e.g. \tlavar{x}, while the values of variables in the
next state use bold script with a prime, e.g. \tlanextvar{x}.  Variables
not mentioned in a formula are unconstrained.

For example, the following formula describes a system where the initial
value of \tlavar{x} is 0 and the value of \tlavar{x} is incremented during
each transition.
\[
\tlavar{x} = 0~\wedge~\Always\left(\tlanextvar{x} = \tlavar{x} + 1\right)
\]
The initial condition ($\tlavar{x} = 0$) is a state formula.  The
transition relation ($\tlanextvar{x} = \tlavar{x} + 1$) is an action
formula and refers to values in the next state using a prime,
e.g. \tlanextvar{x}.  Both the transition relation and the property are
lifted to trace formulas using the always modality ($\Box$).  When always
is applied to an action formula, it means that all pairs of temporally
adjacent states are related by the action formula.  When always is applied
to a state formula, it means that all states satisfy the property.

For convenience, we also use an operator $\Unchanged{X}$, where $X$ is a
set of variables, to represent the LTL formula stating that each variable
in $X$ is equal to its primed counterpart:
\[
\Unchanged{\{\tlavar{x_1},\ldots,\tlavar{x_n}\}} \defined
\tlanextvar{x_1} = \tlavar{x_1}~\wedge~\ldots~\wedge~\tlanextvar{x_n} = \tlavar{x_n}
\]

Finally, the semantics of formulas is defined in terms of two relations:
``models'' (written $tr \models P$) states when a predicate ($P$) holds on
a trace ($tr$), and ``entails'' (written $P \entails Q$, or just $\entails
Q$ when $P$ is trivial) states when one predicate implies another on
\emph{all} traces, i.e.
\begin{definition}[LTL Entailment]
\[\begin{array}{rcl}
P \entails Q & \defined & \forall tr, tr \models P \rightarrow tr \models Q
\end{array}
\]
\label{def:ltl-entails}
\end{definition}
For example, the following states that all traces of the above system have
the property that $\tlavar{x}$ is always at least 0.
\[
\entails \tlavar{x} = 0 \wedge \Always\left(\tlanextvar{x} = \tlavar{x} + 1\right) \rightarrow \Always\left(\tlavar{x} \geq 0\right)
\]
The implication means that the traces of the system are a subset of the
traces for which $\tlavar{x}$ is at least 0 in all states.

\section{Sampled-data systems in LTL}
In a periodic sampled-data system, the state repeatedly transitions either
continuously according to some differential (in)equations or discretely
according to the (possibly nondeterministic) controller.  In addition, the
elapsed time between discrete transitions of the controller is bounded by
some constant.  In LTL, we can model such systems using a formula of the
form
\[
I\wedge\always{(\Sys{D}{\W}{\Delta})}
\]
Here, we use the action formula $\Sys{D}{\W}{\Delta}$, specifying
transitions of a sampled-data system, where: $D$ is an action formula
specifying the discrete controller, and $\W$ is a predicate over state
variables and their derivatives. This can be used to express systems of
differential equations $\dt{\tlavar{x}} = e_1 \wedge \dt{\tlavar{y}} =
e_2$, differential inequalities $\dt{\tlavar{z}} \leq e$, and even pure
state predicates restricting the evolution of variables with respect to
each other $x \leq y$. These expressions can be conjoined to express all
three concepts in the same continuous evolution. Formally,
\begin{definition}[\SysA{} abstraction]
\[\begin{array}{l}
\Sys{D}{\W}{\Delta} \triangleq \\
\qquad
\begin{array}{cl}
& D \wedge \Time{} = 0 \wedge 0 < \tlanextvar{\Time{}} \leq \Delta \\
\vee & \ContinuousP{\left(\W \wedge \dt{\Time{}} = -1\right)} \wedge \tlanextvar{\Time{}} \geq 0 \\
\end{array}
\end{array}
\]
\label{def:sys-abstraction}
\end{definition}
In this action formula, the disjunction captures the fact that the system
transitions either continuously according to the physical world or
discretely according to the controller.  The definition encapsulates both
the semantics of the continuous transition and the timing characteristics
of the system.
%%The two non-trivial aspects of this formula are the specification of
%%continuous transitions and the enforcement of the timing constraint.

Informally, $\Continuous{(\W)}$ means that the state evolves for
\emph{some} amount of time according to a continuous function whose value
and derivative at each point in time satisfy the predicate in \W.
Formally, $\Continuous{(\W)}$ is an LTL action formula, defined as follows:
\begin{definition}[Continuous evolution]
\[\begin{array}{l}
\Continuous{\W} \equiv \\
\quad \exists (r : \R)\, (f : \R \rightarrow \textsf{Var} \rightarrow \R), 0 < r \\
\qquad \wedge\, \forall 0 \leq t \leq r, \W(f(t),\dt{f}(t)) \\
\qquad \wedge\, x_1 = f(0,x_1) \wedge \ldots \wedge x_n = f(0,x_n) \\
\qquad \wedge\, x_1\tlaprime = f(r,x_1) \wedge \ldots \wedge x_n\tlaprime = f(r,x_n)
\end{array}
\]
\label{def:continuous}
\end{definition}
Here, $x_1,\ldots,x_n$ are variables in the system, $r$ is the amount of
time that the system evolves for, and $f$ is a differentiable function from
time to state that gives the evolution of the system state during the
continuous transition. The first conjunct expresses that the predicate \W
holds on the state and its derivative during the entire system evolution.
The second and third conjuncts relate the current state to the value of the
solution $f$ at 0 and the next state to the value of $f$ at time $r$.

At first glance, this definition of continuous transitions may look strange
since it seems to allow the trace to ``skip'' states -- that is, any single
sequence of states satisfying \SysA{} does not capture all intermediate
states of the system. Instead, the discrete trace captures finite
observations along the continuous evolution of the physical world.  In
fact, it may seem as though a sequence of states is a poor fit for
describing the continuously evolving physical world since time and other
continuous variables can only advance in discrete steps. The core of the
argument lies in the fact that in LTL we prove properties of \emph{all}
sequences of states rather than properties of a single sequence of
states. When we prove $\entails \Init\wedge\Sys{D}{\W}{\Delta} \rightarrow
P$ for some property $P$, we are proving that $P$ holds on all sequences of
states that satisfy $\Init\wedge\Sys{D}{\W}{\Delta}$.  In other words, we
are proving properties of all possible sequences of observations of the
hybrid system's state.  While a single trace may skip a certain state
during a continuous transition, another trace does include that state
because the definition of $\Continuous{}$ captures \emph{all} possible
continuous transitions of any duration. The soundness of this encoding is
argued by Lamport in~\cite{lamport2005real}.

Finally, the definition of \SysA{} also expresses that at most $\Delta$
time elapse between executions of the controller. This timing constraint is
captured using the variable \Time{} (not mentioned in $D$ or $\W$), which
tracks the time that elapses between successive transitions of the discrete
controller.  During the continuous evolution of the system, \Time{}
decreases at the same rate as time, i.e. $\dt{\Time{}} = -1$, and
$\tlanextvar{\Time{}} \geq 0$ ensures that no more than $\Delta$ time
elapses between successive discrete transitions of the controller.  The
discrete transition occurs when the timer has expired ($\Time{} = 0$); this
transition resets the timer to a positive value that is at most $\Delta$.

\section{Acknowledgments}
This chapter \prelimack{}
