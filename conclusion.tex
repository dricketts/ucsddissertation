We have presented a series of proof rules for safety verification of
sampled-data systems and evaluated them on various versions of the
sampled-data double integrator, an important benchmark in control
theory. Chapter~\ref{chap:memo15} gave rules for discrete induction
specialized to sampled-data systems and acyclic composition of
components. Chapter~\ref{chap:emsoft16} described a general framework for
building sampled-data systems using conjunction (parallel composition),
disjunction (alternative composition), and substitution (reuse). Finally
Chapter~\ref{chap:exp-smpl} revisited an atomic building block from
previous chapters, using a better performing but more complex position
bounding controller to motivate a stronger continuous induction theorem for
sampled-data systems.

All of the work described in this dissertation was done in the expressive
and foundational context of the Coq proof assistant. This expressiveness
was essential, as it allowed us to add domain-specific reasoning techniques
as theorems and dispatch side conditions using formal proofs, rather than
using axioms or tactics and dispatching side conditions using unverified
decision procedures. In summary, the work described in this document was in
support of the following:

\thesis

There are a number of exciting future research directions for this work. In
the rest of this section, we discuss some of these directions.

\paragraph{Horizontal composition}
Chapter~\ref{chap:emsoft16} explored composition of sampled-data systems,
particularly those controlling overlapping sets of actuators. However, the
proof rules of Chapter~\ref{chap:emsoft16} were only evaluated on the
simpler but poorer performing atomic position (height) controller from
Chapter~\ref{chap:memo15}, not on the more complex but better performing
position controller from Chapter~\ref{chap:exp-smpl}. It would be valuable
to explore composition and re-use of the more complex controller from
Chapter~\ref{chap:exp-smpl}, as this would be an important step towards
verification of the full Ardupilot geofencing module. An promising approach
is to combined ideas from~\cite{xu16sharing}, which explores composition of
barrier certificates for continuous time systems, with theorems from
Chapter~\ref{chap:exp-smpl}, which explore barrier certificates for
sampled-data systems. This might address a particular shortcoming in
Chapter~\ref{chap:emsoft16}: the composition proof rules did not leverage
any of the differential invariants/barrier certificates used to verify
subcomponents.

\paragraph{Vertical composition}
In Section~\ref{sec:exp-smpl-overview}, we described how the real
implementation of the Ardupilot controller does not directly set
acceleration but instead outputs desired velocity commands to an inner loop
velocity controller. To simplify reasoning the outer loop controller was
designed under the approximation that the inner loop controller
instantaneously achieves any desired velocity command. The informal
justification for this approximation is that, as long as the velocity
commands do not change too quickly, the velocity dynamics are sufficiently
fast relative to position dynamics that the approximation error will not
cause significant violations of position constraints. It would be valuable
to formalize this argument: under what conditions can the position
constraint-enforcing outer loop controller be vertically composed with the
inner loop velocity controller, and how much constraint violation does this
cause?

Answering this question will require determining what properties the inner
loop velocity controller must satisfy. The informal argument is based on
the velocity dynamics being sufficiently fast -- this suggests that the
inner loop controller must satisfy a fast convergence property, such as
local exponential stability. It would be valuable to verify such a property
in the same formal framework as this dissertation, suggesting that the Coq
framework needs to be extended with rules for reasoning about stability
properties from control theory, particularly for sampled-data systems.
Another promising approach is to explore the use of reference
governors~\cite{bemporad98reference}, a concept from control theory in an
inner loop controller is augmented with a module (the reference governor)
that modifies commands (references) to the controller in order to guarantee
desired state constraints.

\paragraph{Verified implementations}
All of the systems verified in this dissertation have contained high-level
models of controllers rather than executable code. Our own experience has
demonstrated that bugs can occur at all levels of the development process,
not just in the high-level design. Based on the ideas in
Chapter~\ref{chap:exp-smpl}, we implemented a geofence module for Ardupilot
that prevents a vehicle from leaving a specified safe area, regardless of
what the pilot does. This important safety feature required extending the
one-spatial dimension constraints from Chapter~\ref{chap:exp-smpl} to an
arbitrary 2D polygon and to a circular fence. At the core of this
implementation is a constraint solver that chooses a safe velocity as close
as possible to the pilot's command while satisfying the constraints of each
edge of the fence (or circle). Our original C++ implementation passed the
review process of the Ardupilot developers, was merged into the
development, and run in the current public release of
Ardupilot. Unfortunately, we later discovered that the constraint solving
implementation is incorrect.\footnote{The bug is documented in the
  following github issues:
  \url{https://github.com/ArduPilot/ardupilot/issues/4429} and
  \url{https://github.com/ArduPilot/ardupilot/issues/4807}.}

This experience suggests that it is valuable to have end-to-end formal
verification results, from the high level design all the way to the low
level implementation. Coq is well suited for this task, as it is expressive
enough for both the high level models (as demonstrated in this
dissertation) as well as executable code
verification~\cite{leroy2009compcert,appel2014vst}, even including floating
point arithmetic~\cite{flocq11}. Such an end-to-end verification result
would be challenging for a standalone hybrid system verification tool as it
would need to interface with another tool for formalizing and reasoning
about code. We expect that a shallow encoding of a hybrid system logic in
Coq will facilitate this task, as it will ease the process of integrating
with other Coq developments for reasoning about low level code, such
as~\cite{appel2014vst,flocq11}.

\paragraph{Probability}
All controllers in this dissertation require an estimate of the state of
the system. Since sensors are not perfect, these estimates are inherently
probabilistic. In Chapter~\ref{chap:memo15}, we presented an approach to
reason about systems in the presence of sensor error when there is a
concrete bound on that error. However, this can be overly pessimistic in
certain cases and inaccurate in others. The core sensor fusion algorithms
used to estimate a system's state based on sensor readings (e.g. Kalman
filters) are based on probability~\cite{Simon06estimation}. It would be
valuable to formalize these approaches and then reason about state
constraint enforcing controllers in the presence of probability. Again, the
expressiveness of Coq provides a promising framework for this direction.

\paragraph{Other domains and more examples}
Explore domain-specific proof rules for event-triggered systems and other domains.
More examples
