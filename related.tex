There has been a tremendous amount of work on the specification and
verification of hybrid systems, both from the verification community and
the control theory community.  In this section, we describe some of the
prior work in this area, highlighting the commonalities and difference with
our own work.

\section{Hybrid Automata}
Hybrid automata~\cite{Henzinger96theoryhybrid,Lynch03IO} extend traditional
automata with continuous transitions.  The primary reasoning method for
hybrid automata is model checking~\cite{PHAVerSTTT08,HyTechCAV97}.  There
have been a number of highly successful implementations such as
\textsf{HyTech}~\cite{HyTechCAV97}, \textsf{PHAVer}~\cite{PHAVerSTTT08},
Flow*~\cite{chen2015flow}, and dReach~\cite{kong2015dreach}. While these
are powerful tools, reachability is only decidable for a very restricted
class of automata, for example rectangular automata~\cite{HenzingerKPV98}.
In practice hybrid system model checkers often struggle with large, complex
systems~\cite{muller2000modelling}. As discussed in
Chapter~\ref{chap:emsoft16}, hybrid system model checkers cope with the
complexity by either verifying weaker properties (e.g. bounded safety,
concrete bounds on quantifiers) or by limiting the structure and arithmetic
operators within both discrete and continuous transitions. As noted in that
chapter, the modularity rules presented in that chapter can complement
automated tools by providing a mechanism to decompose a system into
components that are in range for full automation. Making this connection
formal is an interesting direction for future work.

Alur \emph{et al.}~\cite{alur1997modularity} present rules for conjoining
hybrid automata with syntactically independent interface variables and for
renaming variables. Our substitution rules generalize substitution
in~\cite{alur1997modularity} to allow substituting \emph{expressions} for
variables, requiring us to separately justify \progress{} through
invertibility of substitution.  Moreover, different modules often need to
output to the same actuators, and our work pushes beyond the restriction of
syntactically independent variables.  Finally,
unlike~\cite{alur1997modularity}, we present rules for disjunctive
composition.

\section{Deductive logics}
Hybrid systems have been formalized in proof assistants, including the HHL
prover~\cite{LiuHHL10,WangHHL2015} and using Platzer's differential dynamic
logic (\dL{}) in KeYmaera X~\cite{KeYmaeraX}. This section focuses heavily
on \dL{} as it is the most developed logic for hybrid systems
(\ref{sec:related-logic-dl}). We also provide a discussion of HHL and other
work formalizing hybrid systems in general purpose proof assistants
(\ref{sec:related-logic-other}).

\subsection{Differential Dynamic Logic}
\label{sec:related-logic-dl}
The \dL{} logic provides a complete proof
calculus~\cite{Platzer15substitution} for hybrid systems with compositional
proof rules similar to Hoare logic~\cite{hoare1971find}.  It has been
implemented in the KeYmaera X proof assistant~\cite{KeYmaeraX} and applied
to a large number of case studies,
including~\cite{platzercruise11,platzer2009formal,platzerrobots13,Arechiga12refinement}.

The work in Chapter~\ref{chap:memo15} complements the \dL{} logic with an
induction rule specific to sampled-data systems (rather than arbitrary
hybrid systems) as well as an acyclic parallel composition proof rule,
applied to compose sensor error/delay compensation modules with safety
controllers. In addition, that work gives the first formal verification of
a continuous induction rule (i.e. differential
induction~\cite{platzer2010logical,Platzer10DAL}). While the work of this
dissertation was done in a different logic (LTL for
Chapters~\ref{chap:memo15} and~\ref{chap:emsoft16} and a trajectory logic
for Chapter~\ref{chap:exp-smpl}), we believe that all rules can be adapted
to \dL{}. Doing so is an interested direction for future work, as it will
allow a user to reap the benefits of the generality of \dL{} with the
domain specific proof rules of this dissertation, all in a fully formal
context.

The work in Chapter~\ref{chap:emsoft16} further complements \dL{}'s general
proof calculus with rules for building sampled-data systems modularly.
\dL{} is not as expressive as Coq's logic, and some of the proof rules from
Chapter~\ref{chap:emsoft16} are not expressible in \dL{}, namely those with
side conditions (\emph{e.g.}  invertible substitutions, disjointness of
primed variables).  The only way to add these rules to KeYmaera X would be
to extend its core with soundness-critical checking of side conditions or
to employ potentially slow and brittle tactics to prove soundness of
composition on a case-by-case basis.  By working in Coq, we are able to get
higher-order modular \emph{theorems} without compromising soundness.

Since publication of the work in Chapters~\ref{chap:memo15}
and~\ref{chap:emsoft16}, there has been work on embedding \dL{} in the Coq
and Isabelle proof assistants~\cite{Bohrer17-verified-dl}. This is a first
step towards extending \dL{} with the types of domain-specific proof rules
found in this dissertation, without compromising soundness. The Coq and
Isabelle embeddings in ~\cite{Bohrer17-verified-dl} deep embeddings, both
of the logic's syntax and of the proof theory. This was necessary in order
to extract a verified version of the KeYmaera X kernel. On the other hand,
the work in Chapters~\ref{chap:memo15} and~\ref{chap:emsoft16} used a deep
embedding of the syntax but not the proof theory, while the work in
Chapter~\ref{chap:exp-smpl} used a fully shallow embedding of the
logic. Thus, all of the work in this dissertation uses Coq's kernel as the
proof checking core rather than having a separate verified proof checking
kernel. Having a range of embeddings within the same domain could
ultimately permit an evaluation of which embedding is optimal for extending
the proof calculus with higher-order domain-specific proof rules whose side
conditions are not expressible within the object logic (e.g. \dL{}).

After publication of the work in Chapters~\ref{chap:memo15}, there was work
in \dL{} on composition~\cite{Muller16comp,Muller17comp} of hybrid
systems. The authors provide a form of parallel composition of components,
allowing components to interact via ports satisfying input and output
contracts. Unlike the work in Chapter~\ref{chap:emsoft16}, their framework
explicitly disallows components from outputing to the same variables. This
makes the paper most comparable to the composition rule
(Theorem~\ref{thm:sys-compose}) from Chapter~\ref{chap:memo15}. The
framework from~\cite{Muller16comp,Muller17comp} does remove the restriction
to acyclic composition from Theorem~\ref{thm:sys-compose}. However,
parallel composition in that framework is not an associative operation,
while it is in Chapter~\ref{chap:memo15}. This restricts the flexibility of
a user in building and composing more than two modules, as we did in
Section~\ref{sec:delay-comp}. Morever, the work
from~\cite{Muller16comp,Muller17comp} was not done in a higher-order proof
assistant. Thus, the theorem is applied in KeYmaera X using an untrusted
proof tactic. While this does not compromise soundness, it results in
potentially slow and brittle proofs relative to application of a theorem in
a proof assistant like Coq.

Finally, the main application result from Chapter~\ref{chap:exp-smpl}
(Theorem~\ref{thm:x-bound-dbl-int}), while provable in \dL{}, would require
a combination of proof rules including differential induction and
differential auxiliaries to emulate exponential barrier certificates and a
number of discrete rules to handle the discrete transitions introduced by
the sampled-data model. Most significantly, there is no proof rule that
allows the control to be treated as continuous while separately bounding
the error introduced by this approximation. Thus, we complement the general
\dL{} calculus with powerful higher order proof rules that abstract common
and natural reasoning patterns for sampled-data systems. Again, working in
Coq allows us to add these rules as theorems so that we do not compromise
soundness.

\subsection{Other Logics}
\label{sec:related-logic-other}
The HHL prover~\cite{LiuHHL10,WangHHL2015} is an Isabelle/HOL embedding of
Hybrid Hoare logic. Like KeYmaera X, the HHL prover provides a general
proof calculus for hybrid systems but does not contain domain-specific
proof rules like the ones in this dissertation for sampled-data
systems. Morevoer, the embedding does not contain any rules for verifying
invariants of continuous flows. Instead, the prover relies on external
decision procedures that automatically search for invariants. The prover
assumes that invariants returned by these decision procedures are
valid. Thus, formalzing theorems for checking invariants of continuous
flows, as we have done throughout this dissertation, complements the
decision procedures by preventing a bug in the decision procedures from
compromising soundness. Instead, theorems like
Theorem~\ref{thm:barrier-smpl-weak} from Chapter~\ref{chap:exp-smpl} check
the output of decision procedures that produce invariants.

Other work formalizing hybrid systems in general-purpose proof assistants
includes~\cite{volker2000towards,mumm-hybrid-pvs,Niqui2008,GeuversKSW10,Collins10ataylor,anand2015roscoq,arechigainvcuts15,Bohrer17-verified-dl}.
With the exception of~\cite{Bohrer17-verified-dl}, none of this work
includes formalization of \emph{continuous} induction rules like barrier
certificates and differential induction or modular proof rules like the
ones from Chapters~\ref{chap:memo15} and~\ref{chap:emsoft16}. In addition,
the work in Chapter~\ref{chap:exp-smpl} is the first to formalize the
powerful exponential condition form of barrier certificates and the first
to adapt it to sampled-data systems.

\subsection{Temporal Logic}
The foundation for Chapters~\ref{chap:memo15} and~\ref{chap:emsoft16} is
temporal logic, and there has been a lot of work on composition in temporal
logic, most notably by Abadi and
Lamport~\cite{abadi1995conjoin,abadi1994realtime}.  Their work describes
how to reason about the conjunction of LTL specifications, but they do not
deal with the interplay between conjunction and \progress{} or substitution
and \progress{}.  It is important to note that the preservation/\progress{}
split is not the same as the safety/liveness split as both preservation and
progress are safety properties.  Abadi and Lamport address Zeno
specifications in~\cite{abadi1994realtime}, but do not address the
relationship between Zenoness and conjunction.  Finally, neither of these
works addresses substitution in the presence of progress.  It is also
important to note that we use disjunction for non-deterministic choice
between controllers while much of the work in temporal logic uses
disjunction to represent interleaving specifications of asynchronous
concurrent systems.  Other work includes techniques for decomposing LTL
verification into a search for suitable barrier
certificates~\cite{wongpiromsarntemporal15} and deductive rules for
synthesizing controllers satisfying ATL* properties~\cite{dimitrovaATL14}.
While these approaches apply to more temporal logic formulas than ours,
they do not address verification using conjunction, disjunction, and
substitution.  There has also been work on synthesis using approximate
bisimulations~\cite{tabuada08approx}.  We focus on the complementary task
of composing and reusing controllers.

\section{Architectures for Cyber-physical Systems}
In Chapters~\ref{chap:memo15} and~\ref{chap:emsoft16}, we describe how to
architect hybrid systems in order to ensure safety in the presence of
complex, unverified controllers.  There has been prior work in this area,
much of it based on the simplex architecture~\cite{sha1996evolving}.  In
this architecture, there is a simple module that constantly monitors the
system and takes control away from more complex modules before the system
can enter an unsafe state.  We follow a similar principle with our
controller architecture from those two chapters. Our work complements that
based on~\cite{sha1996evolving} by formally verifying the simple monitoring
module (our controllers) rather than relying on their simplicity for
correctness.

Livadas and Lynch solve a similar problem using hybrid I/O automata to
model and reason about ``protectors'' for hybrid systems~\cite{LivadasL98}.
A protector is designed to ensure a safety property of a particular hybrid
system. Livadas and Lynch present a series of rules for conjunctive
composition of protectors.  In contrast, our work also supports other forms
of composition, \textit{e.g.} disjunctive composition, and we show how
these operators can be \emph{combined} to achieve modular verification.

Our system architecture from Chapter~\ref{chap:memo15} is closely related
to the ModelPlex framework in~\cite{mitsch2014modelplex}. This framework
allows a user to produce provably correct runtime monitors for hybrid
systems. These monitors check the execution of a hybrid system
implementation for compliance to a model. If the implementation deviates
from the model, then any safety properties with respect to that model are
no longer guaranteed. In this case, the monitor provides a mechanism for
switching to a fail-safe controller. ModelPlex requires that the safety of
these fail-safe controllers be verified using other techniques. Thus, the
work in this dissertation complements ModelPlex by providing
domain-specific proof rules for verifying fail-safe controllers for
sampled-data systems.

Our theory from Chapter~\ref{chap:emsoft16} is closely related to the work
on modular construction of nonblocking supervisory controllers in
discrete-event systems~\cite{wonham1988supervisory}.  However, our
models explicitly include differential equations rather than requiring a
discrete abstraction and do not require a notion of termination in a marked
state.  Moreover, to the best of our knowledge, none of this work makes the
inductive invariant explicit and separates preservation from \progress{}.

Finally, the area of switching control~\cite{liberzon2012switching} from
control theory focuses on (in our terminology) disjunctive composition of
controllers.  Our inclusion of invariants in the discrete controller
corresponds to expressing the switching boundary in a switching controller.
However, the focus of switching control theory is optimality, stability,
and transitionability, whereas we focus on complementary properties like
bounding the state space.
%TODO read more about switching control

\subsection{Geofences}
There has been research on keeping UAVs in a designated safe area, and more
generally on obstacle avoidance strategies for UAVs. The most related work
is the recent geofencing strategy developed by Gurriet et
al~\cite{gurriet16geofence}. This work places constraints on a pilot's
desired velocity vector to avoid leaving a specified safe region. The
strategy is complementary to the one described in this paper as it leaves
the maximum velocity in a given direction unspecified. Future work can
explore composing their strategy with the concrete velocity constraints
described in Chapter~\ref{chap:exp-smpl}.

\section{Inductive Methods for Continuous Systems}
There is a long history of techniques for establishing forward invariance
of systems of differential equations.\footnote{A set $S$ is forward
  invariant with respect to a system of differential equations if every
  solution starting in $S$ remains in $S$.} One of the earliest such
techniques is known as Nagumo's viability
theorem~\cite{aubin2009viability}. This technique relies on a notion called
the tangent cone of a set at a point. Informally, the tangent cone of a set
$S$ at a point $x$ is the set of all vectors starting from $x$ that point
inward towards $S$. Nagumo's viability theorem states that a closed set $S$
is forward invariance with respect to a system of differential equations
$\dt{\vecbold{x}} = f(\vecbold{x})$, if at all points $\vecbold{x}$ on the
boundary of $S$, $f(\vecbold{x})$ is contained in the tangent cone of
$\vecbold{x}$ at $S$. Such a semantic condition for invariance is powerful,
as it applies to arbitrary closed sets. However, it is difficult to apply
in practice without an effective way of computing the tangent cone of a
particular closed set at a point.

Since Nagumo's theorem, a number of techniques have been developed that
establish forward invariance of particular classes of
sets. In~\cite{prajna04barrier}, the authors coined the term barrier
certificates and showed that for a differentiable function $B$, $B \leq 0$
is forward invariant with respect to a system of differential equations if
the Lie derivative of $B$ along the vector field of that system is
non-positive at $B = 0$.\footnote{The Lie derivative of $B$ along the
  vector field $f(\vecbold{x})$ is defined by $\nabla B \dotprod f$ and
  gives the rate of change along solutions of the system $\dt{\vecbold{x}}
  = f(\vecbold{x})$}. The premise $B=0$ in this rule comes with a number of
drawbacks. First, as the authors note in~\cite{prajna04barrier}, it renders
the set of barrier certificates non-convex, making it inefficient to
automatically search for barrier certificates for a given system. Second,
as is explored in~\cite{Ghorbal17inv}, this premise makes it challenging to
generalize the rule to sets built using finite boolean combinations of
polynomial inequalities. Finally, only considering the Lie derivative of
$B$ exactly at the boundary of the set prevents the rule from being
extended to sampled-data systems using the approach we employ in
Chapter~\ref{chap:exp-smpl}. Because of the first reason, the authors
of~\cite{prajna04barrier} also give a formulation of barrier certificates
without the premise $B = 0$.

Platzer's differential induction~\cite{platzer2010logical,Platzer10DAL}
provides a generalization of this weaker formulation of barrier
certificates (without the premise $B = 0$) to finite boolean combinations
of polynomial inequalities. The total differential operator
from~\cite{platzer2010logical,Platzer10DAL} and the more recent rewrite
rule-based formulation from~\cite{Platzer15substitution} provide an
automatic mechanism for computing Lie-derivatives of finite boolean
combinations of polynomial inequalities. This automation is very useful but
fails to handle piecewise functions as we do in
Chapter~\ref{chap:exp-smpl}. In addition, differential induction does not
generalize the more powerful exponential condition-based barrier
certificates from~\cite{kong2013barrier} that we use in
Chapter~\ref{chap:exp-smpl}. Instead, differential induction must be
augmented with differential auxiliaries~\cite{Platzer12diffcut} to handle
systems that exponentially decay towards the invariant boundary. Most
significantly, the exponential condition-based barrier certificates provide
a natural mechanism for quantifying the invariant violation from a
continuous-time approximation of a sampled-data system, as we do in
Chapter~\ref{chap:exp-smpl}. It is not clear how to do this with
differential induction, as the condition for this proof rule does not imply
that trajectories of the system move towards the invariant region when they
start outside (i.e. after a violation).

Recent stuff on exponential barrier certificates, especially robustness paper by Tabuada and note made at the end

barrier lyapunov functions

Stuff by Tiwaly

HHL stuff

composition of barrier certificates


The only formal implementation of any of these techniques is ...

\subsection{Hybrid Systems}
The generality of the formulations prevents them from...

\section{Sampled-Data Systems}
