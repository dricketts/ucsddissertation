There has been a tremendous amount of work on the specification and
verification of hybrid systems, both from the verification community and
the control theory community.  In this section, we describe some of the
prior work in this area, highlighting the commonalities and difference with
our own work.

\section{Hybrid Automata}

\section{Deductive logics}

\subsection{Temporal Logic}
The foundation for Chapters~\ref{chap:memo15} and~\ref{chap:emsoft16} is
temporal logic, and there has been a lot of work on composition in temporal
logic, most notably by Abadi and
Lamport~\cite{abadi1995conjoin,abadi1994realtime}.  Their work describes
how to reason about the conjunction of LTL specifications, but they do not
deal with the interplay between conjunction and \progress{} or substitution
and \progress{}.  It is important to note that the preservation/\progress{}
split is not the same as the safety/liveness split as both preservation and
progress are safety properties.  Abadi and Lamport address Zeno
specifications in~\cite{abadi1994realtime}, but do not address the
relationship between Zenoness and conjunction.  Finally, neither of these
works addresses substitution in the presence of progress.  It is also
important to note that we use disjunction for non-deterministic choice
between controllers while much of the work in temporal logic uses
disjunction to represent interleaving specifications of asynchronous
concurrent systems.  Other work includes techniques for decomposing LTL
verification into a search for suitable barrier
certificates~\cite{wongpiromsarntemporal15} and deductive rules for
synthesizing controllers satisfying ATL* properties~\cite{dimitrovaATL14}.
While these approaches apply to more temporal logic formulas than ours,
they do not address verification using conjunction, disjunction, and
substitution.  There has also been work on synthesis using approximate
bisimulations~\cite{tabuada08approx}.  We focus on the complementary task
of composing and reusing controllers.

\section{Architectures for Cyber-physical Systems}
In Chapters~\ref{chap:memo15} and~\ref{chap:emsoft16}, we describe how to
architect hybrid systems in order to ensure safety in the presence of
complex, unverified controllers.  There has been prior work in this area,
much of it based on the simplex architecture~\cite{sha1996evolving}.  In
this architecture, there is a simple module that constantly monitors the
system and takes control away from more complex modules before the system
can enter an unsafe state.  We follow a similar principle with our
controller architecture from those two chapters. Our work complements that
based on~\cite{sha1996evolving} by formally verifying the simple monitoring
module (our controllers) rather than relying on their simplicity for
correctness.

Livadas and Lynch solve a similar problem using hybrid I/O automata to
model and reason about ``protectors'' for hybrid systems~\cite{LivadasL98}.
A protector is designed to ensure a safety property of a particular hybrid
system. Livadas and Lynch present a series of rules for conjunctive
composition of protectors.  In contrast, our work also supports other forms
of composition, \textit{e.g.} disjunctive composition, and we show how
these operators can be \emph{combined} to achieve modular verification.

Our theory from Chapter~\ref{chap:emsoft16} is closely related to the work
on modular construction of nonblocking supervisory controllers in
discrete-event systems~\cite{wonham1988supervisory-small}.  However, our
models explicitly include differential equations rather than requiring a
discrete abstraction and do not require a notion of termination in a marked
state.  Moreover, to the best of our knowledge, none of this work makes the
inductive invariant explicit and separates preservation from \progress{}.

Finally, the area of switching control~\cite{liberzon2012switching} from
control theory focuses on (in our terminology) disjunctive composition of
controllers.  Our inclusion of invariants in the discrete controller
corresponds to expressing the switching boundary in a switching controller.
However, the focus of switching control theory is optimality, stability,
and transitionability, whereas we focus on complementary properties like
bounding the state space.
%TODO read more about switching control

\section{Inductive Methods for Hybrid Systems}

\section{Sampled-data Systems}
