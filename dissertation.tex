\documentclass[12pt]{ucsddissertation}
% mathptmx is a Times Roman look-alike (don't use the times package)
% It isn't clear if Times is required. The OGS manual lists several
% "standard fonts" but never says they need to be used.
\usepackage{mathptmx}
\usepackage[NoDate]{currvita}
\usepackage{array}
\usepackage{tabularx}
\usepackage{booktabs}
\usepackage{ragged2e}
\usepackage{microtype}
\usepackage[breaklinks=true,pdfborder={0 0 0}]{hyperref}
\usepackage{graphicx}
\AtBeginDocument{%
	\settowidth\cvlabelwidth{\cvlabelfont 0000--0000}%
}

% OGS recommends increasing the margins slightly.
\increasemargins{.1in}

% These are just for testing/examples, delete them
\usepackage{trace}
%\usepackage{showframe} % This package was just to see page margins
\usepackage[english]{babel}
\usepackage{blindtext}
\overfullrule5pt
% ---

% Required information
\title{Verification of Sampled-Data Systems using Coq}
\author{Daniel Ricketts}
\degree{Computer Science}{Doctor of Philosophy}
% Each member of the committee should be listed as Professor Foo Bar.
% If Professor is not the correct title for one, then titles should be
% omitted entirely.
\chair{Professor Sorin Lerner}
% Your committee members (other than the chairs) must be in alphabetical order
\committee{Professor Samuel Buss}
\committee{Professor William Griswold}
\committee{Professor Ranjit Jhala}
\committee{Professor Todd Millstein}
\degreeyear{2017}

% Start the document
\begin{document}
% Begin with frontmatter and so forth
\frontmatter
\maketitle
\makecopyright
\makesignature
% Optional
\begin{dedication}
\setsinglespacing
\raggedright % It would be better to use \RaggedRight from ragged2e
\parindent0pt\parskip\baselineskip
In recognition of reading this manual before beginning to format the
doctoral dissertation or master's thesis; for following the
instructions written herein; for consulting with OGS Academic Affairs
Advisers; and for not relying on other completed manuscripts, this
manual is dedicated to all graduate students about to complete the
doctoral dissertation or master's thesis.

In recognition that this is my one chance to use whichever
justification, spacing, writing style, text size, and/or textfont that
I want to while still keeping my headings and margins consistent.
\end{dedication}
% Optional
\begin{epigraph}
\vskip0pt plus.5fil
\setsinglespacing
{\flushright
True ease in writing comes from art, not chance,\\
As those move easiest who have learn'd to dance.\\
'T is not enough to no harshness gives offence,---\\
The sound must seem an echo to the sense.

\vskip\baselineskip
\textit{Alexander Pope}\par}
\vfil
\begin{center}
You write with ease to show your breeding,\\
But easy writing's curst hard reading.

\vskip\baselineskip
\textit{Richard Brinsley Sheridan}
\end{center}
\vfil
\noindent Writing, at its best, is a lonely life. Organizations for
writers palliate the writer's loneliness, but I doubt if they improve
his writing. He grows in public stature as he sheds his loneliness and
often his work deteriorates. For he does his work alone and if he is a
good enough writer he must face eternity, or the lack of it, each day.

\vskip\baselineskip
\hskip0pt plus1fil\textit{Ernest Hemingway}\hskip0pt plus4fil\null

\vfil
\end{epigraph}

% Next comes the table of contents, list of figures, list of tables,
% etc. If you have code listings, you can use \listoflistings (or
% \lstlistoflistings) to have it be produced here as well. Same with
% \listofalgorithms.
\tableofcontents
\listoffigures
\listoftables

% Preface
\begin{preface}
Almost nothing is said in the manual about the preface. There is no
indication about how it is to be typeset. Given that, one is forced to
simply typeset it and hope it is accepted. It is, however, optional
and may be omitted.
\end{preface}

% Your fancy acks here. Keep in mind you need to ack each paper you
% use. See the examples here. In addition, each chapter ack needs to
% be repeated at the end of the relevant chapter.
\begin{acknowledgements}
I would like to acknowledge Professor Eta Theta for his support as the
chair of my committee. Through multiple drafts and many long nights,
his guidance has proved to be invaluable.

I would also like to acknowledge the ``Smith Clan'' of lab~28, without
whom my research would have no doubt taken fives times as long. It is
their support that helped me in an immeasureable way.

Chapter 2, in full, is a reprint of the material as it appears in
Numerical Grid Generational in Computational Fluid Mechanics~2009.
Smith, Laura; Smith, Jane~D., Pineridge Press,~2009. The dissertation
author was the primary investigator and author of this paper.

Chapter 3, in part, has been submitted for publication of the material
as it may appear in Education Mechanics,~2009, Smith, Laura; Smith,
Jane~D., Trailor Press,~2009. The dissertation author was the primary
investigator and author of this paper.

Chapter 5, in part is currently being prepared for submission for
publication of the material. Smith, Laura; Smith, Jane~D\@. The
dissertation author was the primary investigator and author of this
material.
\end{acknowledgements}

% Stupid vita goes next
\begin{vita}
\noindent
\begin{cv}{}
\begin{cvlist}{}
\item[1996] Bachelor of Arts, University of California, Berkeley
\item[1996--2000] U.S. Marines
\item[2000--2002] Teaching Assistant, Department of Mechanical
Engineering\\University of California, San Diego
\item[2002--2006] Research Assistant, University of California, San
Diego
\item[2010] Doctor of Philosophy, University of California, San Diego
\end{cvlist}
\end{cv}

% This puts in the PUBLICATIONS header. Note that it appears inside
% the vita environment. It is optional.
\publications
\noindent``Distributions of Control Points in a System for Analysis of Stress
Distribution'' IRE Transactions of the I.R.E\@. Professional Group on
Automatic Control, vol. AC-7, pp 272--289, September 2005

% This puts in the FIELDS OF STUDY. Also inside vita and also
% optional.
\fieldsofstudy
\noindent Major Field: Engineering (Specialization or Focused Studies)
\vskip\baselineskip
Studies in Applied Mathematics\par
Professors Alpha Beta and Gamma Delta
\vskip\baselineskip
Studies in Mechanices\par
Professors Epsilon Zeta and Eta Theta
\vskip\baselineskip
Studies in Electromagnetism\par
Professors Iota Kappa and Lambda Mu
\end{vita}

% Put your maximum 350 word abstract here.
\begin{dissertationabstract}
The Abstract begins here. The abstract is limited to 350 words for a
doctoral dissertation. It should consist of a short statement of the
problem, a brief explanation of the methods and procedures employed in
generating the data, and a condensed summary of the findings of the
study. The abstract may continute onto a second page if necessary. The
text of the abstract must be double spaced.
\end{dissertationabstract}

% This is where the main body of your dissertation goes!
\mainmatter

% Optional Introduction
\begin{dissertationintroduction}
Errors in cyber-physical systems can lead to disastrous consequences,
including loss of life.  These consequences mean that such systems demand
the most rigorous verification techniques.  There has been a variety of
work on developing fully-automated tools for verification of cyber-physical
systems~\cite{PHAVerSTTT08,HyTechCAV97}, but due to the complexity of the
domain, these tools are only able to verify particular classes of systems
and properties.  On the other hand, all cyber-physical systems are in range
for deductive verification in a proof assistant, at least in theory.
However, one of the typically-stated drawbacks of verification in proof
assistants is the extremely high manual labor cost required to produce
these proofs.

Scaling deductive verification to realistic hybrid systems requires
powerful higher-order proof rules that capture common reasoning
patterns. Prior work in deductive verification has produced
general-purpose, complete proof calculi for hybrid
systems~\cite{platzer???,HHL???}. While powerful, the generality of the
proof rules prevents them from leveraging particular common characteristics
of cyber-physical systems, such as their modular construction from smaller
components. This dissertation presents and applies a series of proof rules
that do leverage such characteristics in the domain of \emph{sampled-data
  systems}~\cite{chen1995sampled}. In this important class of systems,
there is a discrete controller that runs periodically.  In between
executions of the controller, the system evolves according to continuous
physical dynamics.

In \textbf{Chapter~\ref{??}} we present two rules: one for verifying a single sampled-data
component under assumption on the environment and another for composing
such a component with another that satisfies this assumption. The first
rule decomposes verification into a property of the discrete controller and
the continuous dynamics, automatically handling the fact that the time
between executions of the controller is bounded. This tailored
decomposition eliminates some of the basic tedious manipulation common to
every sampled-data system, allowing one to focus on the application
specific aspects of verification. The second proof rule allows for
component composition with non-cylical assumptions -- that is, a component
$C1$ can guarantee an invariant $Q$ while assuming and invariant $P$
guaranteed by $C2$. However, $C2$ cannot rely on the invariance of $P$ when
guaranteeing $Q$. In spite of this restriction, we show that such a rule
has important applications for verifying controllers in the presence of
sensor error and delay.

Next, \textbf{Chapter~\ref{??}} presents a general framework for building
sampled-data systems \emph{modularly}. This framework differs from the
above composition approach by requiring that each component provide a
stronger interface. In particular, rather than proving invariance of a
property, each component provides preservation of an inductive invariant,
and a notion of progress of the system under that inductive invariant. This
stronger interface comes at a minor cost while proving two important
benefits. First, it allows for cyclic dependencies between sampled-data
components, thus removing the restriction from Chapter~\ref{???}. Second,
it allows us to explore a richer set of operators for modularly building
and verifying sampled-data components, namely substitution, conjunction,
and disjunction. It is this second benefit that we explore thoroughly by
applying our framework to build verified 3-dimensional geofences for a
UAV. We show that our theory can handle the important situation in which
different components output to the same set of actuators, as exemplified by
the geofence application.

Finally, in \textbf{Chapter~\ref{??}}, we revisit verification of a single
sampled-data system component. Contrary to our prior approaches, we began
by building a geofence that was good enough to be adopted by the popular
open source UAV autopilot called Ardupilot~\cite{???}. After building such
a module (now available in the latest Ardupilot release), we attempted to
formally verify a component of it in Coq, particularly the logic that
prevents the vehicle from violating a boundary in a single spatial
dimension. Similar logic is present in the atomic components from
Chapters~\ref{??}  and~\ref{??}, but realistic performance requirements for
Ardupilot resulted in considerably more complicated control logic. This
additional complexity demanded better proof rules, and we built rules that
improve upon the state of the art in formal verification in two ways.

First, deductive techniques typically involve some continuous analogue of
induction, e.g. differential induction~\cite{Ghorbal14diffinv} or barrier
certificates~\cite{prajna04barrier}. Recent work from the control theory
community~\cite{kong2013barrier,xu15barrier,nguyen16barrier} has produced a
new version of barrier certificates that are less conservative for closed
properties. We provide the first implementation of this approach in a
formal verification context, and demonstrate its ease of use on a component
of the ardupilot geofencing module.

Second, control systems are often designed under the assumption that
controllers run continuously, while the actual implementation is typically
a sampled-data system. System designers can compensate for this (and other)
approximations by adding a safety ``buffer'' to the system. For example,
the ardupilot geofence module stops the vehicle 1 meter prior to the actual
safety boundary. We developed a proof rule that formalizes this design
approach by decomposing verification into a condition on the continuous
time approximation and another on the approximation error.  This rule
allows one to perform the majority of reasoning in a purely continuous
model using powerful techniques resulting from over a century of control
theory research.

Motivated by our attempt to add a geofence to a popular open
source autopilot called Ardupilot~\cite{???}, we

As a realistic running example, we apply all of the rules to.... We justify
the realistic nature of our application by flying the controllers we verify
on an actual quadcopter.

Crucial to the successful implementation of this work is the use of an
expressive proof assistant such as Coq.

\end{dissertationintroduction}

\chapter{An ordinary page}
The purpose of this page is to illustrate an ordinary page of text in
a doctoral dissertation or master's thesis. All pages of the doctoral
dissertation or master's thesis must be kept within the margins of
1.5'' on the left, 1'' on the right, 1'' on the top and 1.25'' on the
bottom. All text must be double spaced except as indicated below.

It is recommended that to increase the margins as paper can shift in a
printer and as some photocopiers tend to increase the image being
copied.

The first line of each paragraph must be indented at least one 0.5''
tab, as done here.

This text is intended to be a part of the dissertation, for a doctoral
student, or the thesis if you are receiving a master's degree, and now
a quote is included here:
\begin{quote}
All quotes of more than six lines, even though this one is not, are to
be indented 0.5'' from the left and 0.5'' from the right. These longer
quotes are to be single spaced. Don't forget to adjust for proper
spacing after the last line of the quoted material.
\end{quote}
The rest of the paragraph would continue as so.

\chapter{Figures and Such}
This demonstrates how OGS wants figures and tables formatted. For
figures, the caption goes below the figure and ``Figure'' is in bold.
See Figure~\ref{fig:zen}. Tables are formatted with the caption above
the table. See Table~\ref{tab:bad}.

Of course, Table~\ref{tab:bad} looks horrible. It should probably be
formatted like Table~\ref{tab:good} instead.

For facing caption pages, see Table~\ref{tab:facing}. Of course,
facing caption pages are vaguely ridiculous and my implementation of
them in the class file is by far the most brittle part of the
implementation. It's entirely possible that something has changed and
these don't work at all. I implemented it merely for the challenge.

\begin{figure}
\centering
\fbox{\parbox{.9\linewidth}{%
	\noindent
	{\Huge PHD ZEN}\par
	\vskip.5in
	\centerline{comic here}
	\vskip.5in
}}
\caption[``Ph.D. Zen'']{Comic entitled ``Ph.D. Zen'' by Jorge Cham, 2005. Copyright
has not been obtained and so it isn't displayed.}
\label{fig:zen}
\end{figure}

\begin{table}
\centering
\caption[Electronic Dissertation Submission Rates]{Electronic
Dissertation Submission Rates at UCSD, Fall 2005 and Winter 2006.
(First two quarters that the program was available to all Ph.D.
candidates not in a Joint Doctoral Program with SDSU.)}
\label{tab:bad}
\begin{tabular}{|*{5}{>{\centering\arraybackslash}m{.15\linewidth}|}}
\hline
&Ph.D.s awarded (Including Joint degrees) & Electronic submission of
Dissertation & Paper Submission of Dissertation & Percentage of
Electronic Submission\\
\hline
Fall\par 2005 & 84 & 37 & 47 & 44.05\%\\
\hline
Winter\par 2006 & 64 & 42 & 22 & 65.63\%\\
\hline
\end{tabular}
\end{table}

\begin{table}
\centering
\caption[Electronic Dissertation Submission Rates]{Electronic
Dissertation Submission Rates at UCSD, Fall 2005 and Winter 2006.
(First two quarters that the program was available to all Ph.D.
candidates not in a Joint Doctoral Program with SDSU.)}
\label{tab:good}
\renewcommand\tabularxcolumn[1]{>{\RaggedRight\arraybackslash}p{#1}}
\begin{tabularx}{.9\linewidth}{lcccc}
\toprule
&\multicolumn{1}{X}{Ph.D.s awarded (Including Joint degrees)}
&\multicolumn{1}{X}{Electronic submission of Dissertation}
&\multicolumn{1}{X}{Paper Submission of Dissertation}
&\multicolumn{1}{X}{Percentage of Electronic Submission}\\
\midrule
Fall 2005 & 84 & 37 & 47 & 44.05\%\\
Winter 2006 & 64 & 42 & 22 & 65.63\%\\
\bottomrule
\end{tabularx}
\end{table}

\begin{facingcaption}{table}
\caption[UCSD Gender Distribution]{University of
California, San Diego Gender Distribution for the Campus Population,
October~2005\\
(http://assp.ucsd.edu/analytical/Campus\%20Population.shtml)\\
\emph{(This is an example of a facing caption page, the next page is
the example of the table/figure/etc.\ that corresponds to this
caption. It is also an example of table/figure that is rotated 90
degrees to fit the page.)}}
\label{tab:facing}
\renewcommand\tabularxcolumn[1]{>{\RaggedLeft\arraybackslash}p{#1}}
\parindent=0pt
\setbox0=\vbox}
& \multicolumn{1}{c}{\textbf{N}} & \multicolumn{1}{c}{\textbf{\%}}
& \multicolumn{1}{c}{\textbf{N}} & \multicolumn{1}{c}{\textbf{\%}}\\
\midrule
Students & 12,987 & 51\% & 12,686 & 49\% & 25,673 & 100\%\\
Employees & 9,943 & 56\% &  7,671 & 44\% & 17,614 & 100\%\\
\addlinespace
\hfill\textbf{Total} & \textbf{22,930} & \textbf{53\%} &
\textbf{20,357} & \textbf{47\%} & \textbf{43,287} & \textbf{100\%}\\
\bottomrule
\end{tabularx}
\singlespacing

\emph{Notes}:
\begin{enumerate}
\item The counts shown below will differ from the official quarterly
Registrar's registration report because 1) data for residents in the
Schools of Medicine and Pharmacy and Pharmaceutical Science are
excluded, and 2) registered, non-matriculated, visiting students are
included.
\item Student workers are excluded from employees; however emeritus
faculty and others on recall status are included.
\end{enumerate}

Campus Planning. Analytical Studies and Space Planning\\
31 January 2006
}
\centerline{\rotatebox{90}{\box0}}
\end{facingcaption}

% This will give us some more text
\Blinddocument

% Skipping a bunch of chapters
\setcounter{chapter}{50}
\chapter{Another chapter}
\setcounter{figure}{73}
\setcounter{table}{88}
\begin{figure}
\centering
\fbox{\hbox to.8\linewidth{\hss Another figure\hss}}
\caption{Another figure caption}
\end{figure}
\begin{table}
\centering
\caption{Another table caption}
\begin{tabular}{ccc}
\toprule
X&Y&Z\\
\midrule
a&b&c\\
\bottomrule
\end{tabular}
\end{table}
\begin{figure}
\caption{ASDF fig}
\end{figure}
\begin{table}
\caption{ASDF tab}
\end{table}

\appendix
\Blinddocument

% Stuff at the end of the dissertation goes in the back matter
\backmatter
\bibliographystyle{plain} % Or whatever style you want like plainnat
\bibliography{references}

\end{document}
