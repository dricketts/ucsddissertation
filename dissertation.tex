\documentclass[12pt]{ucsddissertation}
% mathptmx is a Times Roman look-alike (don't use the times package)
% It isn't clear if Times is required. The OGS manual lists several
% "standard fonts" but never says they need to be used.
\usepackage{mathptmx}
\usepackage[NoDate]{currvita}
\usepackage{array}
\usepackage{tabularx}
\usepackage{booktabs}
\usepackage{ragged2e}
\usepackage{microtype}
\usepackage[breaklinks=true,pdfborder={0 0 0}]{hyperref}
\usepackage{graphicx}
\AtBeginDocument{%
	\settowidth\cvlabelwidth{\cvlabelfont 0000--0000}%
}

% OGS recommends increasing the margins slightly.
\increasemargins{.1in}

% These are just for testing/examples, delete them
\usepackage{trace}
%\usepackage{showframe} % This package was just to see page margins
\usepackage[english]{babel}
\overfullrule5pt
% ---

% Packages specific to this dissertation
\usepackage{amsmath}
\usepackage{amssymb}
\usepackage{contour}
\usepackage{amsthm}
\usepackage{xcolor}
\usepackage{bussproofs}
\usepackage{listings}
\usepackage{hyperref}
\usepackage{tabularx}
\usepackage{tikz}
\usepackage{macros}
\usepackage{xspace}
\usepackage{adjustbox}
%\usepackage{gensymb}
%\usepackage{pgfplots}
\usepackage{etex}
\usepackage{todonotes}
\usepackage{mfirstuc}
\usepackage{changebar}
\usepackage{subcaption}
\usepackage{enumitem}
\usepackage{thmtools}
\usepackage{thm-restate}
\usepackage{bibentry}
\nobibliography*

\usetikzlibrary{calc}
\usetikzlibrary{backgrounds}
\usetikzlibrary{patterns}
\usetikzlibrary{decorations}
\usetikzlibrary{arrows}
\usetikzlibrary{decorations.markings}
\usetikzlibrary{positioning}

\declaretheorem[name=Theorem,numberwithin=chapter]{theorem}
\newtheorem{lemma}{Lemma}[chapter]
\declaretheorem[name=Definition,numberwithin=chapter]{definition}
\newtheorem{assumption}{Assumption}[chapter]

% Required information
\title{Verification of Sampled-Data Systems using Coq}
\author{Daniel Ricketts}
\degree{Computer Science}{Doctor of Philosophy}
% Each member of the committee should be listed as Professor Foo Bar.
% If Professor is not the correct title for one, then titles should be
% omitted entirely.
\chair{Professor Sorin Lerner}
% Your committee members (other than the chairs) must be in alphabetical order
\committee{Professor Samuel Buss}
\committee{Professor William Griswold}
\committee{Professor Ranjit Jhala}
\committee{Professor Todd Millstein}
\degreeyear{2017}

% Start the document
\begin{document}
% Begin with frontmatter and so forth
\frontmatter
\maketitle
\makecopyright
\makesignature
% Optional
\begin{dedication}
\setsinglespacing
\raggedright % It would be better to use \RaggedRight from ragged2e
\parindent0pt\parskip\baselineskip
\centering
\emph{To my wife, parents, and brother, for always believing in me}
\end{dedication}

% Next comes the table of contents, list of figures, list of tables,
% etc. If you have code listings, you can use \listoflistings (or
% \lstlistoflistings) to have it be produced here as well. Same with
% \listofalgorithms.
\tableofcontents
\listoffigures
\listoftables

% Your fancy acks here. Keep in mind you need to ack each paper you
% use. See the examples here. In addition, each chapter ack needs to
% be repeated at the end of the relevant chapter.
\begin{acknowledgements}
I owe a tremendous debt to my adviser Sorin Lerner for supporting me both
technically and otherwise throughout a long PhD. From taking me on as an
untested student switching research areas to letting me spend half of grad
school in Germany with my wife, he has always put my interests first.

A special thanks also goes to my original adviser Mohan Paturi, who always
gave me the freedom to pursue what I enjoyed and supported me when
switching to a new research area and adviser.

It was a great experience working with Gregory Malecha, who showed me the
importance of technical simplicity and abstraction and who convinced me
that everything is an adjoint functor. Don Jang's always positive mood and
ability to finish any task with an impending deadline are an inspiration.
Zachary ``Danger'' Tatlock's excitement for research, outgoing nature, and
ability to bounce back from any setback are a model to emulate, as are his
poncho-based outfits. Valentin Robert helped introduce me to the world of
verification and always kept a relaxed attitude, no matter the situation.

I very much enjoyed working with the Ardupilot developers, particularly
Leonard Hall, who always had time for a fun and enlightening conversation.

One of my earliest collaborators, Moshe Hoffman, showed me what passion for
research looks like, while Petros Mol was always there to re mind me that
there is more to life than work.

I want to thank Wilson Lian for going through le tour de grad school with
me, for all the deep thought-filled conversations, the trap chair-based
entertainment, and for always being a supportive and understanding friend.

I also want to thank Matt Der for being an honest and enthusiastic friend,
and for always prioritizing soccer training over research so that we may
guide the US to a World Cup victory.

I could not have done this without my parents, who taught me to persevere
through all challenges, and my brother Scott, who always has my back and
showed me that you should never worry about what others think.

And most importantly, I want to thank my wife Maja for being a never ending
source of love and support through the highs and lows of Coq, for never
letting me get too frustrated, and for always being proud of me no matter
what.

Chapter~\ref{chap:memo15}, \memoack{}

Chapter~\ref{chap:emsoft16}, \emsoftack{}

Chapter~\ref{chap:exp-smpl}, \expsmplack{}

Chapter~\ref{chap:related}, includes and expands upon material selected
from the above works.
\end{acknowledgements}

% Stupid vita goes next
\begin{vita}
\noindent
\begin{cv}{}
\begin{cvlist}{}
\item[2009] Bachelor of Science, Brown University
\item[2009-2010] Software Engineer, Inria
\item[2011] Intern, Telef\'onica Research
\item[2012] Intern, Facebook
\item[2010--2017] Research Assistant, University of California, San
Diego
\item[2017] Doctor of Philosophy, University of California, San Diego
\end{cvlist}
\end{cv}

% This puts in the PUBLICATIONS header. Note that it appears inside
% the vita environment. It is optional.
\publications
\noindent\bibentry{ricketts2016composition}
\\ \\
\noindent\bibentry{ricketts2015memocode}
\\ \\
\noindent\bibentry{ricketts14reflex}
\\ \\
\noindent\bibentry{cousineau2012tla+}
\\ \\
\noindent\bibentry{dalkiran2012common}

\end{vita}

% Put your maximum 350 word abstract here.
\begin{dissertationabstract}
Due to their safety-critical nature, cyber-physical systems (CPS) demand
the most rigorous verification techniques. However, the complexity of the
domain puts many cyber-physical systems outside the scope of automated
verification tools. Formal deductive proofs hold the potential to verify
virtually any property of any system, but proofs for practical
cyber-physical systems often require an impractical amount of manual
effort. This proof burden can be mitigated by capturing common reasoning
patterns in powerful higher-order proof rules. Existing work has focused on
proof rules applicable to arbitrary hybrid systems (a formal model for
CPS), but many systems actually fall into more constrained classes. One
such class of systems are called sampled-data systems, in which a discrete
controller runs periodically. In this dissertation, we complement general
hybrid system proof rules with a series of rules that leverage the
particular structure of sampled-data systems. We demonstrate the
applicability of these rules on the double integrator, an important model
in robotic and vehicle systems. All work is formalized in the Coq proof
assistant, whose expressive logic is crucial to maintaining soundness while
applying domain-specific proof rules for sampled-data systems. Finally, we
experimentally evaluate our results by implementing verified controllers on
a quadcopter and conducting flight tests.
\end{dissertationabstract}

% This is where the main body of your dissertation goes!
\mainmatter

% Optional Introduction
\begin{dissertationintroduction}
Errors in cyber-physical systems (CPS) can lead to disastrous consequences,
including loss of life.  These consequences mean that such systems demand
the most rigorous verification techniques.  There has been a variety of
work on developing fully-automated tools for verification of cyber-physical
systems~\cite{PHAVerSTTT08,HyTechCAV97}, but due to the complexity of the
domain, these tools are only able to verify particular classes of systems
and properties.  On the other hand, all cyber-physical systems are in range
for deductive verification in a proof assistant, at least in theory.  In
this technique, a user writes a formal model of a system in the language of
the proof assistant and then interactively proves it correct. However, one
of the typically-stated drawbacks of verification in proof assistants is
the extremely high manual labor cost required to produce these proofs.

Mitigating this manual proof burden requires powerful higher-order proof
rules that capture common proof strategies. Prior work in deductive
verification has approached this task by designing general-purpose,
complete proof calculi for hybrid
systems~\cite{Platzer15substitution,LiuHHL10,WangHHL2015}. Hybrid systems
are CPS models comprised of both a discrete component (e.g. control
software) and a continuous component (the physical world). While powerful,
the generality of the proof calculi prevents them from leveraging
particular common characteristics of cyber-physical systems. For example,
no proof rule in~\cite{Platzer15substitution} assumes that the time between
executions of a controller is bounded because this is not true of all
hybrid systems. However, this is true of many realistic hybrid systems, and
proofs about such systems will tend to follow a similar proof
structure. Thus, it is beneficial to complement general hybrid system proof
rules with domain specific proof rules that capture common reasoning
patterns.

This dissertation presents and applies a series of proof rules that capture
common reasoning patterns in the important domain of \emph{periodic
  sampled-data systems}~\cite{chen1995sampled}. In such a system, there is
a digital controller that runs periodically. In between executions of the
controller, the system evolves according to continuous physical
dynamics. Many modern cyber-physical systems fit into this domain. The
remainder of the introduction provides an overview of the proof rules. In
general, the rules seek to leverage timing characteristics of systems and
improve modularity of reasoning.

In \textbf{Chapter~\ref{chap:memo15}} we present two rules: one for
verifying a single sampled-data component under assumption on the
environment and another for composing such a component with another that
satisfies this assumption. The first rule decomposes verification into a
property of the discrete controller and the continuous dynamics,
automatically handling the fact that the time between executions of the
controller is bounded. This tailored decomposition eliminates some of the
basic tedious manipulation common to every sampled-data system, allowing
one to focus on the application specific aspects of verification. The
second proof rule allows for component composition with non-cyclical
assumptions -- that is, a component $C1$ can guarantee an invariant $Q$
while assuming and invariant $P$ guaranteed by $C2$. However, $C2$ cannot
rely on the invariance of $P$ when guaranteeing $Q$. In spite of this
restriction, we show that such a rule has important applications for
verifying controllers in the presence of sensor error and delay.

Next, \textbf{Chapter~\ref{chap:emsoft16}} presents a general framework for
building sampled-data systems \emph{modularly}. This framework differs from
the above composition approach by requiring that each component provide a
stronger interface. In particular, rather than proving invariance of a
property, each component provides preservation of an inductive invariant,
and a notion of progress of the system under that inductive invariant. This
stronger interface comes at a minor cost while proving two important
benefits. First, it allows for cyclic dependencies between sampled-data
components, thus removing the restriction from
Chapter~\ref{chap:memo15}. Second, it allows us to explore a richer set of
operators for modularly building and verifying sampled-data components,
namely substitution, conjunction, and disjunction. It is this second
benefit that we explore thoroughly by applying our framework to build
verified 3-dimensional geofences for a UAV. We show that our theory can
handle the important situation in which different components output to the
same set of actuators, as exemplified by the geofence application.

Finally, in \textbf{Chapter~\ref{chap:exp-smpl}}, we revisit verification
of a single sampled-data system component. Contrary to our prior
approaches, we began by building a geofence that was good enough to be
adopted by the popular open source UAV autopilot called
Ardupilot~\cite{ardupilot}. After building such a module (now available in the
latest Ardupilot release), we attempted to formally verify a component of
it in Coq, particularly the logic that prevents the vehicle from violating
a boundary in a single spatial dimension. Similar logic is present in the
atomic components from Chapters~\ref{chap:memo15} and~\ref{chap:emsoft16},
but realistic performance requirements for Ardupilot resulted in
considerably more complicated control logic. This additional complexity
demanded better proof rules, and we built rules that improve upon the state
of the art in formal verification in two ways.

First, deductive techniques typically involve some continuous analogue of
induction, e.g. differential induction~\cite{Platzer10DAL} or barrier
certificates~\cite{prajna04barrier}. Recent work from the control theory
community~\cite{kong2013barrier,xu15barrier,nguyen16barrier} has produced a
new version of barrier certificates that are less conservative for closed
properties. We provide the first implementation of this approach in a
formal verification context, and demonstrate its ease of use on a component
of the Ardupilot geofencing module.

Second, control systems are often designed under the assumption that
controllers run continuously, while the actual implementation is typically
a sampled-data system. System designers can compensate for this (and other)
approximations by adding a safety ``buffer'' to the system. For example,
the Ardupilot geofence module stops the vehicle 1 meter prior to the actual
safety boundary. We developed a proof rule that formalizes this design
approach by decomposing verification into a condition on the continuous
time approximation and another on the approximation error.  This rule
allows one to perform the majority of reasoning in a purely continuous
model using powerful techniques resulting from over a century of control
theory research~\cite{kong2013barrier,xu15barrier,nguyen16barrier}.

As already mentioned, our running application in this work is a geofencing
controller for UAVs, an important application due to their potential safety
threat combined with widespread use by hobbyists and businesses alike. We
formally model such systems using the double integrator from control
theory, in which the controller affects the acceleration (second
derivative) of position. This is an approximation of reality, as the actual
quadcopter controller affects the angular torque of the vehicle. The
angular torque is related to the second derivative of position via complex
attitude dynamics. Although our model is an approximation, the double
integrator is a benchmark problem for robotics, vehicles, and other systems
that involve a point mass moving in space~\cite{rao2001naive}. Thus, the
double integrator serves as a canonical example for evaluating the
expressiveness of the proof rules presented in this dissertation.

Since we use an approximate model of UAV dynamics, we would like to ensure
that the controllers we build and verify are not toys. We have attempted to
justify the realistic nature of our work by flying the controllers we
verify on an actual quadcopter. Throughout this dissertation, we discuss
lessons learned from this experience.

Rigorous verification requires that results be mechanically checked in some
way. Rather than implementing a standalone tool for this task, we performed
all verification within the foundational Coq proof assistant. Previous
work~\cite{yang2011understanding-compiler-bugs} has empirically
demonstrated that foundationally verified systems are highly
reliable. While important, we would like to emphasize a less discussed
benefit of higher-order proof assistants like Coq. A standalone domain
specific verification tool would require adding each proof rule as an
axiom. Moreover, a new proof rule might have side conditions not
expressible in the tool's logic, requiring custom reasoning to handle these
side conditions. In this context, the axiom and associated custom reasoning
have the potential to compromise soundness.

On the other hand, the \emph{expressiveness} of Coq allows one add new
proof rules as theorems that are formally proven within Coq's logic. This
means that one can extend any given set of general proof rules in Coq
(e.g. general proof calculi for hybrid systems) with powerful domain
specific proof rules (e.g. our sampled-data system rules), without
compromising soundness. Such an extension improves verification
productivity by ensuring that a user can apply the right domain-specific
proof rule for his or her application while still being able to depend on
the above mentioned reliability guarantees.
\thesis
It is thus the view of the dissertation author that the expressive
framework of a higher-order proof assistant is crucial to scaling formal
verification to realistic cyber-physical systems, and we hope that the
proof rules and associated applications in this document provide evidence
for this claim.

\end{dissertationintroduction}

\chapter{Preliminaries}
\label{chap:prelim}
In this chapter, we describe the logical framework used in
Chapters~\ref{chap:memo15}
and~\ref{chap:emsoft16}. Chapter~\ref{chap:exp-smpl} uses a different
framework, so the logical background is presented within that chapter. All
of our work is formalized inside the Coq proof assistant, but for
exposition purposes, we focus on the mathematical concepts rather than
concrete Coq syntax.

\section{Linear Temporal Logic}
In Chapters~\ref{chap:memo15} and~\ref{chap:emsoft16}, we encode
sampled-data systems and their properties within discrete-time linear
temporal logic (LTL).  An LTL formula specifies the possible traces of a
system.  In our model, a trace is an infinite sequence of states
representing observations of a system at discrete points in time.  A state
is a mapping from variables to real numbers.  Inspired by Lamport's
Temporal Logic of Actions (TLA)~\cite{lamport1994temporal}, formulas in our
logic are classified into \emph{state formulas} (predicates over a single
state), \emph{action formulas} (state relations specifying system
transitions), and \emph{trace formulas} (predicates over traces).  In
action formulas, the values of variables in the current state are notated
using bold script, e.g. \tlavar{x}, while the values of variables in the
next state use bold script with a prime, e.g. \tlanextvar{x}.  Variables
not mentioned in a formula are unconstrained.

For example, the following formula describes a system where the initial
value of \tlavar{x} is 0 and the value of \tlavar{x} is incremented during
each transition.
\[
\tlavar{x} = 0~\wedge~\Always\left(\tlanextvar{x} = \tlavar{x} + 1\right)
\]
The initial condition ($\tlavar{x} = 0$) is a state formula.  The
transition relation ($\tlanextvar{x} = \tlavar{x} + 1$) is an action
formula and refers to values in the next state using a prime,
e.g. \tlanextvar{x}.  Both the transition relation and the property are
lifted to trace formulas using the always modality ($\Box$).  When always
is applied to an action formula, it means that all pairs of temporally
adjacent states are related by the action formula.  When always is applied
to a state formula, it means that all states satisfy the property.

For convenience, we also use an operator $\Unchanged{X}$, where $X$ is a
set of variables, to represent the LTL formula stating that each variable
in $X$ is equal to its primed counterpart:
\[
\Unchanged{\{\tlavar{x_1},\ldots,\tlavar{x_n}\}} \defined
\tlanextvar{x_1} = \tlavar{x_1}~\wedge~\ldots~\wedge~\tlanextvar{x_n} = \tlavar{x_n}
\]

Finally, the semantics of formulas is defined in terms of two relations:
``models'' (written $tr \models P$) states when a predicate ($P$) holds on
a trace ($tr$), and ``entails'' (written $P \entails Q$, or just $\entails
Q$ when $P$ is trivial) states when one predicate implies another on
\emph{all} traces, i.e.
\begin{definition}[LTL Entailment]
\[\begin{array}{rcl}
P \entails Q & \defined & \forall tr, tr \models P \rightarrow tr \models Q
\end{array}
\]
\label{def:ltl-entails}
\end{definition}
For example, the following states that all traces of the above system have
the property that $\tlavar{x}$ is always at least 0.
\[
\entails \tlavar{x} = 0 \wedge \Always\left(\tlanextvar{x} = \tlavar{x} + 1\right) \rightarrow \Always\left(\tlavar{x} \geq 0\right)
\]
The implication means that the traces of the system are a subset of the
traces for which $\tlavar{x}$ is at least 0 in all states.

\section{Sampled-data systems in LTL}
In a periodic sampled-data system, the state repeatedly transitions either
continuously according to some differential (in)equations or discretely
according to the (possibly nondeterministic) controller.  In addition, the
elapsed time between discrete transitions of the controller is bounded by
some constant.  In LTL, we can model such systems using a formula of the
form
\[
I\wedge\always{(\Sys{D}{\W}{\Delta})}
\]
Here, we use the action formula $\Sys{D}{\W}{\Delta}$, specifying
transitions of a sampled-data system, where: $D$ is an action formula
specifying the discrete controller, and $\W$ is a predicate over state
variables and their derivatives. This can be used to express systems of
differential equations $\dt{\tlavar{x}} = e_1 \wedge \dt{\tlavar{y}} =
e_2$, differential inequalities $\dt{\tlavar{z}} \leq e$, and even pure
state predicates restricting the evolution of variables with respect to
each other $x \leq y$. These expressions can be conjoined to express all
three concepts in the same continuous evolution. Formally,
\begin{definition}[\SysA{} abstraction]
\[\begin{array}{l}
\Sys{D}{\W}{\Delta} \triangleq \\
\qquad
\begin{array}{cl}
& D \wedge \Time{} = 0 \wedge 0 < \tlanextvar{\Time{}} \leq \Delta \\
\vee & \ContinuousP{\left(\W \wedge \dt{\Time{}} = -1\right)} \wedge \tlanextvar{\Time{}} \geq 0 \\
\end{array}
\end{array}
\]
\label{def:sys-abstraction}
\end{definition}
In this action formula, the disjunction captures the fact that the system
transitions either continuously according to the physical world or
discretely according to the controller.  The definition encapsulates both
the semantics of the continuous transition and the timing characteristics
of the system.
%%The two non-trivial aspects of this formula are the specification of
%%continuous transitions and the enforcement of the timing constraint.

Informally, $\Continuous{(\W)}$ means that the state evolves for
\emph{some} amount of time according to a continuous function whose value
and derivative at each point in time satisfy the predicate in \W.
Formally, $\Continuous{(\W)}$ is an LTL action formula, defined as follows:
\begin{definition}[Continuous evolution]
\[\begin{array}{l}
\Continuous{\W} \equiv \\
\quad \exists (r : \R)\, (f : \R \rightarrow \textsf{Var} \rightarrow \R), 0 < r \\
\qquad \wedge\, \forall 0 \leq t \leq r, \W(f(t),\dt{f}(t)) \\
\qquad \wedge\, x_1 = f(0,x_1) \wedge \ldots \wedge x_n = f(0,x_n) \\
\qquad \wedge\, x_1\tlaprime = f(r,x_1) \wedge \ldots \wedge x_n\tlaprime = f(r,x_n)
\end{array}
\]
\label{def:continuous}
\end{definition}
Here, $x_1,\ldots,x_n$ are variables in the system, $r$ is the amount of
time that the system evolves for, and $f$ is a differentiable function from
time to state that gives the evolution of the system state during the
continuous transition. The first conjunct expresses that the predicate \W
holds on the state and its derivative during the entire system evolution.
The second and third conjuncts relate the current state to the value of the
solution $f$ at 0 and the next state to the value of $f$ at time $r$.

At first glance, this definition of continuous transitions may look strange
since it seems to allow the trace to ``skip'' states -- that is, any single
sequence of states satisfying \SysA{} does not capture all intermediate
states of the system. Instead, the discrete trace captures finite
observations along the continuous evolution of the physical world.  In
fact, it may seem as though a sequence of states is a poor fit for
describing the continuously evolving physical world since time and other
continuous variables can only advance in discrete steps. The core of the
argument lies in the fact that in LTL we prove properties of \emph{all}
sequences of states rather than properties of a single sequence of
states. When we prove $\entails \Init\wedge\Sys{D}{\W}{\Delta} \rightarrow
P$ for some property $P$, we are proving that $P$ holds on all sequences of
states that satisfy $\Init\wedge\Sys{D}{\W}{\Delta}$.  In other words, we
are proving properties of all possible sequences of observations of the
hybrid system's state.  While a single trace may skip a certain state
during a continuous transition, another trace does include that state
because the definition of $\Continuous{}$ captures \emph{all} possible
continuous transitions of any duration. The soundness of this encoding is
argued by Lamport in~\cite{lamport2005real}.

Finally, the definition of \SysA{} also expresses that at most $\Delta$
time elapse between executions of the controller. This timing constraint is
captured using the variable \Time{} (not mentioned in $D$ or $\W$), which
tracks the time that elapses between successive transitions of the discrete
controller.  During the continuous evolution of the system, \Time{}
decreases at the same rate as time, i.e. $\dt{\Time{}} = -1$, and
$\tlanextvar{\Time{}} \geq 0$ ensures that no more than $\Delta$ time
elapses between successive discrete transitions of the controller.  The
discrete transition occurs when the timer has expired ($\Time{} = 0$); this
transition resets the timer to a positive value that is at most $\Delta$.

\section{Acknowledgments}
This chapter \prelimack{}


\chapter{Discrete Induction and Basic Composition}
\label{chap:memo15}
This chapter presents two proof rules for reasoning about sampled-data
systems. The first rule is an induction rule specialized to sampled-data
systems. This rule decomposes an inductive safety proof into a proof about
the initial state, a proof about the discrete transition, and a proof about
the continuous transition. For each of the transitions, the proof rule
introduces timing constraints that are guaranteed by the periodic
sampled-data model. Such a rule automatically manages the aspects of an
inductive safety proof that are common to all periodic sampled-data
systems, allowing the user to focus on the application specific aspects of
the proof. Without such a proof rule, this tedious decomposition would have
to be done manually.

The second rule provides a simple mechanism for composing systems with
non-cyclic dependencies. This allows one component to assume that invariant
of another when proving its own invariant. We demonstrate how this can be
used to reason about systems in the presence of both sensor error and delay
by chaining instances of this rule together. Chapter~\ref{chap:emsoft16}
presents an alternate approach that removes the non-cyclic restriction.

We introduce the proof rules along side a running example: a controller
that prevents a quadcopter from violating some maximum velocity.  We
additionally used the rules to verify a controller that prevents a
quadcopter from violating some maximum height, and used our composition
rule to verify them both in the presence of sensor error and delay.

In an effort to ensure that our examples are realistic, we implemented them
on a actual quadcopter and performed flight tests. Doing so forced us to
confront an important issue. Since our controllers are components of a much
larger system containing numerous controllers, how do we architect the
autopilot to incorporate our controllers? Throughout the paper, we describe
our solution and discuss engineering tradeoffs. This solution has not been
formally verified. That is, we have not formally modeled the entire
autopilot and verified the properties of our controllers in that
context. We view this as an important direction for future work. We also
discuss discrepancies between our model and reality as well as lessons
learned from doing foundational verification of sampled-data systems.



\chapter{Modular Verification}
\label{chap:emsoft16}
Chapter~\ref{chap:memo15} introduced two atomic controllers that enforced
desired safety properties when run separately. That chapter also described
how to compose components, but informally followed the convention that
components being composed do not constrain overlapping sets of
variables. However, such a restriction is natural in sampled-data systems,
for example when composing two controllers outputting to the same sets of
actuators. This raises the question: when can two such controllers be run
in parallel in order to enforce the conjunction of their respective safety
properties? More generally, when can a sampled-data system be built up
modularly from smaller components while ensuring the properties guaranteed
by each of the components and avoiding inconsistencies in the composed
system? This chapter provides a series of proof rules to address that
question.

As alluded to above, one of the challenges is in ensuring that the
specification of the discrete controller is always enabled, i.e. it always
specifies at least one successor state.  While this might seem trivial,
consider the following scenario.  Suppose we have built a module that
prevents a quadcopter from exceeding some maximum altitude.  Furthermore,
suppose we have also built another module that prevents the quadcopter from
violating some \emph{minimum} altitude.  If we have separately verified
that these two modules enforce their respective properties, we would like
to compose them in parallel to guarantee both properties simultaneously.
That is, we would like the composed system to guarantee that the system
never goes too high or too low.  However, this is not always possible; a
module could enforce the upper bound on altitude by always accelerating
downwards.  Likewise, a module could enforce the lower bound on altitude by
always accelerating upwards.  Clearly, na\"ively composing the controllers
of these modules in parallel would result in a system that gets stuck --
there is never an acceleration that both controllers can agree on.

In this chapter, we present sound techniques for resolving this and other
potential pitfalls for reusing and composing modules for sampled-data
systems.  We observed that modularity is facilitated by separating
verification into two parts: \emph{preservation} and \emph{\progress{}}.
Preservation ensures that the model guarantees the safety property
inductively, while \progress{} ensures that the system model is always
enabled.  This separation facilitates the application of several simple,
general, and powerful operators, namely substitution, conjunction, and
disjunction.  More precisely, we state sufficient conditions for applying
these operators to individual modules to produce a new sampled-data system
with the desired properties (e.g. the conjunction of the safety properties
of conjoined modules).  Crucially, these sufficient conditions are in terms
of preservation and \progress{}.

To validate the expressiveness of our theorems, as in the rest of the
dissertation we apply them in the context of \emph{quadcopters}, by showing
how to compose several simple verified controllers together in different
ways to produce many different verified composed controllers.  To ensure
that our controllers are practical, we run them on an actual quadcopter.
In summary, the contributions of this chapter are:
\begin{itemize}
\setlength\itemsep{0.01em}
\item We implement in the Coq proof assistant a general approach for modular verification of sampled-data systems by separately exposing proofs of preservation and \progress{}. The development is available from: \url{https://github.com/ucsd-pl/veridrone/tree/EMSOFT-16}.
\item We apply this approach to build and verify arbitrary 3D geofences for a quadcopter, including walls, boxes, and rectangular donuts, starting with two simple verified 1D controllers.
We show that our modular verification techniques keep the proof burden manageable.
\item We evaluate our geofences by running them on an actual quadcopter, and show that they work in practice.
\item We discuss the capabilities of three state-of-the-art fully-automated tools (SpaceEx, Flow*, and dReach) in verifying our geofence controllers.
\end{itemize}

\section{Overview}
We start with an informal description of the operators that our theory
covers: substitution, conjunction, and disjunction.  We then give an
overview of verifying controllers using these operators.  Finally, we
describe how we applied this to build a verified family of geofences for a
quadcopter.

\paragraph*{Operators}
\emph{Substitution} of expressions for variables represents a form of
reuse, allowing us to transform systems and their properties into a
different coordinate system.  For example, given a model of a system
defined in the x-y plane, the substitution $\{\tlavar{x} \mapsto \tlavar{r}
\cos \tlavar{\roll},\; \tlavar{y} \mapsto \tlavar{r} \sin \tlavar{\roll}\}$
transforms the model to polar coordinates, the substitution $\{\tlavar{x}
\mapsto \tlavar{y},\; \tlavar{y} \mapsto \tlavar{x}\}$ rotates the system,
and the substitution $\{\tlavar{x} \mapsto \tlavar{x} + 5\}$ translates the
system by 5 units in the $\tlavar{x}$ dimension.  However, not all
substitutions soundly transport both systems \emph{and} their properties;
our theory (Section~\ref{sec:subst}) gives formal conditions under which
substitutions are permitted.

\emph{Disjunction} of two systems represents nondeterministic choice
between the controllers of the system.  For example, if we have a
controller that prevents a quadcopter from flying too far to the west and
another controller that prevents a quadcopter from flying too far north,
then their disjunction enforces a north-west no-fly zone -- the quadcopter
must stay to the north \emph{or} to the west of the no-fly zone.  Unlike
conjunction, there is no risk of the composed system getting stuck.
Instead, the challenge with disjunction is in constraining the
nondeterministic choice between the controllers.  Our theorems and
definitions in Section~\ref{sec:disjunctive-composition} make this formal
by including the inductive invariants of each system within the composed
controller.

\emph{Conjunction} of two systems represents parallel composition of these
systems.  For example, if we have a system that enforces an upper bound on
velocity and a system that enforces a lower bound on velocity, then their
conjunction enforces both an upper and a lower bound on velocity.  We can
also conjoin systems that control or restrict different variables, such as
a system that enforces a bound on velocity and a system that enforces a
bound on position.  However, as discussed in the introduction, the
challenge of applying this operator is in ensuring that the conjoined
systems never get stuck, e.g. when the controller of one system requires
positive acceleration while the other requires negative acceleration.
Again, our theory (Section~\ref{sec:conjunctive-composition}) gives formal
conditions under which conjunctive composition is possible.

Note that disjunctive and conjunctive composition are related to
alternative and parallel composition.

\paragraph*{Controller Verification}
To illustrate how the operators work, we explain the construction and
modular verification of several general purpose controllers for enforcing
state constraints, depicted in Figure~\ref{fig:library}.  We begin with a
simple verified module: a controller that enforces an upper bound on
position in one spatial dimension by controlling acceleration (depicted by
(a) in Figure~\ref{fig:library}).  We use substitution to ``mirror'' this
module and its correctness property, thus obtaining a module (b) that
enforces a lower bound on position, again in one dimension.  We conjoin
these two modules to form a controller (c) enforcing upper and lower bounds
on position, still in one dimension.  We use substitution to rotate this
interval controller into a second, orthogonal dimension (d), then conjoin
(c) and (d) to form a controller (e) enforcing a 2 dimensional rectangle,
i.e. upper and lower bounds on position in two dimensions.  Finally, we use
substitution to build and verify four translated copies of (e) and
disjunction of these four copies to enforce a rectangular donut (f).  We
use disjunction to enforce that the system must, at all times, be in the
first copy of (e), the second, the third, \emph{or} the fourth.  Moreover,
since the rectangles are overlapping, the system can transition from one
rectangle to another.

\begin{figure}[t]
  \centering
  \begin{tikzpicture}
    \tikzstyle{every node}=[fill=black!15,node distance=0.5cm,font=\scriptsize]
    \tikzstyle{every path}=[draw=black,ultra thick]

   \node[draw=none,minimum height=0.65cm,minimum width=0.65cm,label=left:{(a)}] (y+) at (-2,2) {}
     (\tikzlastnode.north west)edge(\tikzlastnode.north east);

   \node[draw=none,minimum height=0.65cm,minimum width=0.65cm,label=right:{(b)}] (y-) at (2,2) {}
     (\tikzlastnode.south west)edge(\tikzlastnode.south east);

   \def\aoff{0.1};

   \draw[-latex,thick] ($(y+.east) + (\aoff,0)$) -- ($(y-.west) - (\aoff,0)$)
     node[draw=none,fill=none,font=\scriptsize,midway,above] {substitution};

   \def\coffx{0.35};
   \def\coffy{0.45};

   \node[draw=none,minimum height=0.65cm,minimum width=0.65cm,label=left:{(c)}] (y+-) at (-2,0.55) {}
     (\tikzlastnode.north west)edge(\tikzlastnode.north east)
     (\tikzlastnode.south west)edge(\tikzlastnode.south east)
     node[fill=none] at ($(y+-.north east) +(\coffx,\coffy)+(0.1,0)$) {conjunction};

   \draw[-latex,thick] ($(y+.south)-(0,\aoff)$) -- ($(y+-.north) + (0,\aoff)$);
   \draw[-latex,thick] (y-) -- (y+-);

   \node[draw=none,minimum height=0.65cm,minimum width=0.65cm,label=right:{(d)}] (x+-) at (2,0.55) {}
     (\tikzlastnode.north west)edge(\tikzlastnode.south west)
     (\tikzlastnode.north east)edge(\tikzlastnode.south east);

   \draw[-latex,thick] ($(y+-.east) + (\aoff,0)$) -- ($(x+-.west) - (\aoff,0)$)
     node[draw=none,fill=none,font=\scriptsize,midway,above] {substitution};

   \node[draw=black,ultra thick,minimum height=0.65cm,minimum width=0.65cm,label=left:{(e)}] (xy+-) at (-2,-1) {}
     node[fill=none] at ($(xy+-.north east) +(\coffx,\coffy)+(0.1,0)$) {conjunction};

   \draw[-latex,thick] ($(y+-.south) - (0,\aoff)$) -- ($(xy+-.north) + (0,\aoff)$);
   \draw[-latex,thick] (x+-) -- (xy+-);

   \node[draw=black,ultra thick,minimum height=1.5cm,minimum width=1.5cm,label=right:{(f)}] (donut_out) at (2,-1) {};
   \node[draw=black,ultra thick,fill=white,minimum height=0.5cm,minimum width=0.5cm] (donut_in) at (2,-1) {};
   \node[draw=black,thick,dashed,fill=none,minimum height=0.5cm,minimum width=1.5cm] (donut1) at (2,-0.5) {};
   \node[draw=black,thick,dashed,fill=none,minimum height=0.5cm,minimum width=1.5cm] (donut2) at (2,-1.5) {};
   \node[draw=black,thick,dashed,fill=none,minimum height=1.5cm,minimum width=0.5cm] (donut3) at (2.5,-1) {};
   \node[draw=black,thick,dashed,fill=none,minimum height=1.5cm,minimum width=0.5cm] (donut4) at (1.5,-1) {};

   \draw[-latex,thick] (xy+-.east) -- (donut1.center);
   \draw[-latex,thick] (xy+-) -- (donut2.center)
     node[draw=none,fill=none,font=\scriptsize,pos=0.4,below,text width=2cm] {substitution and disjunction};
   \draw[-latex,thick] (xy+-) edge[thick,out=0,in=170] (donut3.center);
   \draw[-latex,thick] (xy+-) -- (donut4.center);

  \end{tikzpicture}

     \caption{Overview of construction and verification of position bounding controllers.}
     \label{fig:library}
\end{figure}

\paragraph*{Quadcopter}
Although the above approach can enforce state constraints for a variety of
applications (e.g. trains, intelligent cruise control), we evaluate our
approach in the context of \emph{quadcopter} controllers that enforce
position and velocity bounds.  We performed this verification modularly
starting from two simple verified modules that we ported from prior work:
one enforcing an upper bound on velocity and another enforcing an upper
bound on position, both in one spatial dimension.  Connecting the
verification methodology above to quadcopters required application of the
three operators under the complex, coupled dynamics of a quadcopter, thus
showcasing the applicability of our rules to solve complex problems.
Crucially, our Coq theorems for each of these operators give formal
conditions under which this is sound.

Ultimately, we were able to use the verification techniques in
Figure~\ref{fig:library} to build a three dimensional bounding box of both
position and velocity for the quadcopter.  This bounding cube provides a
powerful building block for constructing ``pixelated shapes'' (analogous to
(f) in Figure~\ref{fig:library}), which can be used to enforce interesting
shapes such as a flying around and over but not near the pilot.  The
results of our verification along with actual flight tests are in
Section~\ref{sec:eval}.  A primary takeaway is that substitution,
conjunction, and disjunction are powerful operations that can take simple
controllers verified with respect to simple dynamics and turn them into
controllers that enforce complex constraints in complex dynamics.

\section{A Modular Basis for Reasoning}
\label{sec:compositional-basis}
In this section, we present the framework that we use for modular reasoning
about sampled-data systems: separating proofs into preservation and
\progress{}.  This foundation will allow us to build the theory for
applying substitution, conjunction, and disjunction (presented in
Section~\ref{sec:compositional-monitors}).  We start by formally
illustrating the difficulty of modular reasoning in this domain. We use the
notation and definitions from Chapter~\ref{chap:prelim}. As a small note,
throughout this section when $X$ is a system, we use $\D_X$ and $\W_X$ to
denote the discrete and continuous transitions of $X$, respectively.  Also,
we use the inductive invariant of a system as its initial condition; thus
we use $I$ to denote an inductive invariant and $I_X$ to denote the
inductive invariant of system $X$.  In practice, one must prove that the
initial condition of a system implies the inductive invariant.

\subsection{Stuck Specifications}
The physical world always evolves because time always evolves.
Cyber-physical system specifications should adhere to this property -- the
specification should never reach a state in which it is stuck, i.e. in
which a transition is impossible.  For example, consider the system
$\Sys{\mathsf{False}}{\W}{\Delta}$.  In this system, there is never a
discrete transition (expressed using the unsatisfiable action formula
$\mathsf{False}$).  Since a discrete transition never occurs, a continuous
transition is not possible once time reaches $\Delta$.  Readers familiar
with Zeno specifications~\cite{abadi1994realtime} will note that \SysA{}
specifications that are never stuck are non-Zeno.

We rule out such specifications using a new abstraction called \System,
defined as follows:
\[
\SystemP{D}{\W}{\Delta} \triangleq \Sys{D}{\W}{\Delta} \vee \neg \EnabledP{\big(\Sys{D}{\W}{\Delta}\big)}
\]
In the above, \Enabled takes an action formula and returns a state formula.
In particular, $\Enabled (A)$ holds on a given state $st$ iff there exists
a next state $st'$ such that $(st, st')\in A$, i.e. the system can take an
$A$ transition.  A specification whose transition is built using \System
can never become stuck; if the underlying \SysA{} becomes stuck (not
\Enabled), then the clause $\neg \EnabledP{\big(\Sys{D}{\W}{\Delta}\big)}$
conservatively expresses that anything can happen.  Informally, we will not
be able to prove any interesting global properties of a \System when the
underlying \SysA{} can reach a state in which it is not \Enabled since we
will know nothing about the next state.

It may seem trivial to avoid writing stuck specifications for sampled-data
systems, and thus the distinction between \SysA{} and \System appears to be
only theoretical.  However, avoiding stuck specifications is a core
challenge of building sampled-data systems modularly. To see why, consider
the following. In general, our end goal is to prove properties of the form:
\[
\entails I \wedge \Always (\SystemP{D}{\W}{\Delta}) \rightarrow \Always S
\]
This property states that, starting with initial condition $I$, condition
$S$ always holds if at each point in the trace, the transition relation is
described by $(\SystemP{D}{\W}{\Delta})$.  Unfortunately, properties like
the one above are not modular.  For example, suppose we have two discrete
transitions $D_1$ and $D_2$ which independently ensure $S_1$ and $S_2$,
i.e.
\[
\vdash I \wedge \Always (\SystemP{D_1}{\W}{\Delta}) \rightarrow \Always S_1
\]
\[
\vdash I \wedge \Always (\SystemP{D_2}{\W}{\Delta}) \rightarrow \Always S_2
\]
We would like to combine these proofs to show that $S_1~\wedge~S_2$ is an
invariant of the conjoined system \SystemP{(D_1 \wedge D_2)}{\W}{\Delta}.
Unfolding the definition of \System reveals that this is not, in general,
true.  The problem is that $\EnabledP{D_1} \wedge \EnabledP{D_2}$ does not
necessarily imply $\EnabledP{(D_1 \wedge D_2)}$ (Consider
\EnabledP{(\tlavar{x}\tlaprime = 1~\wedge~\tlavar{x}\tlaprime = 0)}).
This means that the following two formulas are not necessarily equivalent:
\[
\SystemP{D_1}{\W}{\Delta}\wedge\SystemP{D_2}{\W}{\Delta}
\]
\[
\SystemP{(D_1 \wedge D_2)}{\W}{\Delta}
\]
This formalizes the challenge described in the introduction -- na\"ive
parallel composition (conjunction) of controllers can result in a
controller that gets stuck.

Crucially, \Enabled is inherently non-modular, so global invariant proofs
of systems specified using \System are inherently non-modular. By ruling
out stuck specifications, we also rule out the modularity of \emph{global}
invariant proofs.

\subsection{Regaining Modularity}
The key to regaining modularity is a shift from global proofs to local
ones.  In particular, we will make the inductive invariant of the system
explicit and use it to prove two properties independently: preservation of
the invariant, and progress of the system under the invariant.  As we will
see in Section~\ref{sec:compositional-monitors}, this decomposition of the
global property into local ones makes it much easier to combine and re-use
systems and their proofs.

\paragraph{Preservation}
Preservation of property ($I$) under an action formula states if $I$ holds
in the current state then it holds in the next state. Formally,
\[
\SysPreservesP{I}{\big(\SysP{D}{\W}{\Delta}\big)} \;\triangleq\; I \wedge \mathsf{Sys}_{\mathsf{inv}} \wedge \SysP{D}{\W}{\Delta} \rightarrow I\tlaprime
\]
where $I\tlaprime$ represents the state formula $I$ with all variables
primed.  The $\mathsf{Sys}_{\mathsf{inv}}$ premise expresses the invariants
guaranteed by the \SysA{} abstraction, namely that no more than $\Delta$
time elapses between discrete transitions.

\paragraph{\Progress{}}
Progress under an invariant justifies that the system is \Enabled assuming
the invariant. Formally,
\[\begin{array}{l}
\SysNeverStuckP{I}{\left(\Sys{D}{\W}{\Delta}\right)} \triangleq \\
\qquad I \wedge \mathsf{Sys}_{\mathsf{inv}} \rightarrow \EnabledP{\left(\Sys{D}{\W}{\Delta}\right)}
\end{array}
\]
This condition allows us to prove that a \SysA{} and a \System{} describe
exactly the same system.

Note that, here, progress is a safety property that is closely related to
the notion of progress in programming languages.  It is different than
progress properties in distributed systems, and it is different than
convergence to an equilibrium in control theory.

\paragraph{Combining Preservation \& \Progress{}}
Preservation of and \progress{} under the \emph{same} inductive invariant
is sufficient to prove that the invariant is a global invariant of the
corresponding \System, which is ultimately our goal.  This is captured by
the following theorem:

\begin{theorem}{\ProofRule{LocalToGlobal}}
\[\begin{array}{rl}
 & \SysPreservesP{I}{\left(\Sys{D}{\W}{\Delta}\right)} \\
\wedge & \SysNeverStuckP{I}{\left(\Sys{D}{\W}{\Delta}\right)} \\
\wedge & I \rightarrow S \\
\vdash & I \wedge \Always \left(\SystemP{D}{\W}{\Delta}\right) \rightarrow \Always S
\end{array}
\]
\end{theorem}

\section{Modular Sampled-data Systems}
\label{sec:compositional-monitors}

In this section, we show how to use preservation and \progress{} to reason
modularly about sampled-data systems.  In particular, for each of our three
operators (substitution, disjunction, and conjunction), we present theorems
that state formal conditions under which application of the operator
guarantees preservation and \progress{}.  We illustrate each of the
operators and corresponding theorems by building verified
state-constraining controllers for quadcopters.  This allows us to
construct and verify controllers enforcing policies such as ``do not fly
above 400 feet'' (FAA regulation for recreational drones), ``do not fly
within 5 miles of an airport'', and ``do not fly within 5 feet of the
pilot.''

It is important to note that all of the state-constraining controllers that
we verify are \emph{non-deterministic}.  This means that the discrete
transitions do not compute a single value for each control variable
(e.g. acceleration) but instead describe a set of allowed values that
ensure the desired state-constraint.  As we will see, this non-determinism
is crucial for conjunctive composition
(Section~\ref{sec:conjunctive-composition}).  In Section~\ref{sec:eval}, we
discuss how the actual implementation of these controllers resolves this
non-determinism.

\paragraph{Building blocks}
As the basic building blocks of our development, we start with the two
sampled-data systems that are minor variants on the ones from
Chapter~\ref{chap:memo15}. Both operate in one spatial dimension, i.e.
\[\begin{array}{rcl}
\W_{1D} & \triangleq & \dt{\tlavar{y}} = \tlavar{v} \wedge \dt{\tlavar{v}} = \tlavar{a} \wedge \dt{\tlavar{a}} = 0
\end{array}
\]
The two controllers each enforce constant bounds on a state variable by
controlling acceleration (\tlavar{a}).  The first-derivative controller
(\derivShimv) bounds velocity using acceleration (the first-derivative of
velocity).  The second-derivative controller (\derivShimx) bounds position
using acceleration (the second-derivative of position).  To ensure that
\derivShimx{} can stop before violating the boundary, the controller is
parameterized by \Tmin which represents the braking acceleration and
smallest possible acceleration (and is negative).
Figure~\ref{fig:formal-monitors} gives the discrete transitions and
inductive invariants for the two systems. Each inductive invariant states
that, given the time until the next discrete transition, the system can
stop before the boundary.


\begin{figure}[t]

\textbf{First-Derivative Controller} ($\derivShimv = \Sys{D_\partial}{W_{1D}}{\Delta}$)
\[\begin{array}{rl}
D_\partial \triangleq & \Ca\tlaprime \wedge (\tlanextvar{a} \cdot \Delta + \tlavar{v} \leq ub \vee \tlanextvar{a} \leq 0) \\
\invShimv \triangleq & (\tlavar{a} < 0 \rightarrow \tlavar{v} \leq ub) \wedge
(\tlavar{a} \geq 0 \rightarrow \tlavar{a} \cdot \Time + \tlavar{v} \leq ub) \\
\end{array}
\]

\textbf{Second-Derivative Controller} ($\derivShimx = \Sys{D_{\partial^2}}{W_{1D}}{\Delta}$)
\[
\begin{array}{rl}
D_{\partial^2} \triangleq & (0 \leq \tlavar{v} + \tlanextvar{a} \cdot \Delta \rightarrow \\
& \;\quad \tdist{\tlavar{v}}{\tlanextvar{a}}{\Delta} + \sdist{\tlavar{v} + \tlanextvar{a}\cdot\Delta} + \tlavar{y} \leq \ubY) \\
& (\tlavar{v} + \tlanextvar{a} \cdot \Delta \leq 0 \wedge 0 < \tlavar{v} \rightarrow \\
& \;\quad \tdist{\tlavar{v}}{\tlanextvar{a}}{\frac{-\tlavar{v}}{\tlanextvar{a}}} + \tlavar{y} \leq \ubY) \wedge \Ca\tlaprime \\
\invShimx \triangleq & \forall t : \mathbb{R}, 0 \leq t \leq \Time \rightarrow \\
& \;\quad \tlavar{\Y} {+} \tdist{\tlavar{\V}}{\tlavar{a}}{t} + \sdist {\max(0,\tlavar{\V} + \tlavar{a} \cdot t)} \leq \ubY
\end{array}
\]
\[
\begin{array}{ccc}
\tdist{\V}{a}{\Delta} \triangleq \V\cdot\Delta+\frac{a\cdot\Delta^2}{2} & \qquad &
\sdist{\V} \triangleq -\frac{\V^2}{2\cdot a_{\mathsf{min}}} \\
\Ca \triangleq \amin \leq \tlavar{a} & \qquad & \amin  < 0
\end{array}
\]

\caption{Discrete transitions and inductive invariants for our building blocks.}
\label{fig:formal-monitors}
\end{figure}

In Chapter~\ref{chap:memo15}, we verified variants on both of these
controllers in a global style but did \emph{not} compose them.  For the
work in this chapter, we ported each of the global proofs to our local,
modular specification by extracting the inductive invariant (which was
stated explicitly in the proof) and the preservation proof (which formed
the inductive case).  Beyond extracting the safety proofs, we also had to
verify \progress{}, which was not addressed in Chapter~\ref{chap:memo15},
but is trivial for such basic modules.  In the remainder of this section,
we denote the preservation and \progress{} proofs of the two controllers
by: \ProofRule{$\partial$-Preserves},
\ProofRule{$\partial$-\xmakefirstuc{\progress{}}},
\ProofRule{$\partial^2$-Preserves}, and
\ProofRule{$\partial^2$-\xmakefirstuc{\progress{}}}.

\paragraph{Quadcopter Controller}
\label{sec:quadcopter-dynamics}
To build and verify controllers for quadcopters, we need a model of the
physical dynamics of a quadcopter, called $\W_{QC}$:
\[\begin{array}{rl}
\W_{QC} \triangleq & \left(\begin{array}{ll}
       & \dt{\tlavar{x}} = \tlavar{v_x} \wedge \dt{\tlavar{y}} = \tlavar{v_y} \wedge \dt{\tlavar{z}} = \tlavar{v_z} \\
\wedge & \dt{\tlavar{v_x}} = \tlavar{T} \cos \tlavar{\roll} \sin \tlavar{\pitch} \\
\wedge & \dt{\tlavar{v_y}} = -\tlavar{T} \sin \tlavar{\roll} \\
\wedge & \dt{\tlavar{v_z}} = \tlavar{T} \cos \tlavar{\roll} \cos \tlavar{\pitch} - g \\
\wedge & \dt{\tlavar{\roll}} = 0 \wedge \dt{\tlavar{\pitch}} = 0 \wedge \dt{\tlavar{T}} = 0
\end{array}\right)
\end{array}
\]

\begin{figure}
\centering

\newcommand{\rotateRPY}[4][0/0/0]% point to be saved to \savedxyz, roll, pitch, yaw
{   \pgfmathsetmacro{\rollangle}{#2}
   \pgfmathsetmacro{\pitchangle}{#3}
   \pgfmathsetmacro{\yawangle}{#4}

   % to what vector is the x unit vector transformed, and which 2D vector is this?
   \pgfmathsetmacro{\newxx}{cos(\yawangle)*cos(\pitchangle)}% a
   \pgfmathsetmacro{\newxy}{sin(\yawangle)*cos(\pitchangle)}% d
   \pgfmathsetmacro{\newxz}{-sin(\pitchangle)}% g
   \path (\newxx,\newxy,\newxz);
   \pgfgetlastxy{\nxx}{\nxy};

   % to what vector is the y unit vector transformed, and which 2D vector is this?
   \pgfmathsetmacro{\newyx}{cos(\yawangle)*sin(\pitchangle)*sin(\rollangle)-sin(\yawangle)*cos(\rollangle)}% b
   \pgfmathsetmacro{\newyy}{sin(\yawangle)*sin(\pitchangle)*sin(\rollangle)+ cos(\yawangle)*cos(\rollangle)}% e
   \pgfmathsetmacro{\newyz}{cos(\pitchangle)*sin(\rollangle)}% h
   \path (\newyx,\newyy,\newyz);
   \pgfgetlastxy{\nyx}{\nyy};

   % to what vector is the z unit vector transformed, and which 2D vector is this?
   \pgfmathsetmacro{\newzx}{cos(\yawangle)*sin(\pitchangle)*cos(\rollangle)+ sin(\yawangle)*sin(\rollangle)}
   \pgfmathsetmacro{\newzy}{sin(\yawangle)*sin(\pitchangle)*cos(\rollangle)-cos(\yawangle)*sin(\rollangle)}
   \pgfmathsetmacro{\newzz}{cos(\pitchangle)*cos(\rollangle)}
   \path (\newzx,\newzy,\newzz);
   \pgfgetlastxy{\nzx}{\nzy};

   % transform the point given by #1
   \foreach \x/\y/\z in {#1}
   {   \pgfmathsetmacro{\transformedx}{\x*\newxx+\y*\newyx+\z*\newzx}
       \pgfmathsetmacro{\transformedy}{\x*\newxy+\y*\newyy+\z*\newzy}
       \pgfmathsetmacro{\transformedz}{\x*\newxz+\y*\newyz+\z*\newzz}
       \xdef\savedx{\transformedx}
       \xdef\savedy{\transformedy}
       \xdef\savedz{\transformedz}
   }
}

\tikzset{RPY/.style={x={(\nxx,\nxy)},y={(\nyx,\nyy)},z={(\nzx,\nzy)}}}

\begin{tikzpicture}

 \begin{scope}[yshift=-0.75cm]
 \draw[-latex,dashed] (-0.3,0,0) to node[near end,above] {$x$} (1,0,0) node[right] {\footnotesize pitch axis (\tlavar{\pitch})} ;
 \draw[-latex,dashed] (0,-0.3,0) to node[near end,right] {$z$} (0,1,0) ;
 \draw[-latex,dashed] (0,0,-0.3) to node[near end,left] {$y$} (0,0,1) node[below] {\footnotesize roll axis (\tlavar{\roll})} ;
 \node[anchor=east] at (-0.1,0.3,0) {\footnotesize ``Earth frame''} ;
 \end{scope}

 \rotateRPY{30}{-20}{0}

 \begin{scope}[xshift=4cm,yshift=-0.5cm]

   \draw[dotted] (-0.3,0,0) -- (0.5,0,0) ;
   \draw[dotted] (0,-0.3,0) -- (0,0.5,0) ;
   \draw[dotted] (0,0,-0.3) -- (0,0,0.5) ;

   \begin{scope}[RPY]
   \draw (0,0,0) to (0.5,0,0) ;
   \draw[-latex] (0,0,0) to node[near end,left] {\tlavar{T}} (0,0.5,0) ;
   \draw (0,0,0) to (0,0,0.5) ;
   \fill[black] (0,0,0) circle (0.05cm) ;
   \end{scope}

   \draw[latex-,gray] (0,0,0) to[bend right] (0.5,0.5) node[inner sep=0.1,anchor=south,black] {\footnotesize quadcopter} ; 

   \draw[-latex] (0,0,0) to node[near end,right] {$g$} (0,-1,0) ;
 \end{scope}

\end{tikzpicture}


\caption{Free-body diagram and dynamics of the quadcopter.}
\label{fig:free-body}
\end{figure}

Here \tlavar{T} represents the combined thrust of the motors (normalized
with respect to the mass of the quadcopter), \tlavar{\pitch} represents the
pitch (the angle around the $y$-axis), and \tlavar{\roll} represents the
roll (the angle around the $x$-axis). Figure~\ref{fig:free-body} depicts
this graphically in a free-body diagram. Our model is based on the
simplifying assumption (called the ``small angle condition'') that a
trusted attitude controller can instantaneously achieve any pitch and roll
within the bounds $-30\mydegree$ to $30\mydegree$ with a thrust greater
than or equal to 0, while holding yaw constant at 0.
\[
\smallangle \triangleq \left| \tlavar{\pitch} \right| \leq 30\mydegree \wedge \left|\tlavar{\roll}\right| \leq 30\mydegree \wedge 0 \leq \tlavar{T}
\]
Prior work has suggested that this is a reasonable approximation under this
small-angle condition ($\smallangle$), since the attitude dynamics are
significantly faster than the velocity and position
dynamics~\cite{Gillula2011}.  We capture this condition by requiring that
all quadcopter controllers are enabled under $\smallangle$.  That is, our
goal is to build controllers $D$ such that
\[
\begin{array}{l}
\SysPreservesP{I}{(\Sys{(D\wedge\Next{\smallangle})}{\W_{QC}}{\Delta})}~\wedge \\
\SysNeverStuckP{I}{(\Sys{(D\wedge\Next{\smallangle}))}{\W_{QC}}{\Delta})}
\end{array}
\]

For some state-constraints and their corresponding controllers, it is only
necessary to reason about an abstraction of the quadcopter dynamics
$\W_{QC}$.  For example, reasoning about a controller that enforces a bound
on the vertical position \tlavar{z} might only require reasoning about the
portion of the dynamics on which \tlavar{z} depends.  We formalize this
with:
\[\begin{array}{rl}
 & \SysPreservesP{I}{(\Sys{(D\wedge\Next{\smallangle}))}{\W}{\Delta})}~\wedge \\
 & \SysNeverStuckP{I}{(\Sys{(D\wedge\Next{\smallangle})}{\W}{\Delta})} \\
 & (\W_{QC} \rightarrow \W)~\wedge \\
\entails & \SysPreservesP{I}{(\Sys{(D\wedge\Next{\smallangle})}{\W_{QC}}{\Delta})}~\wedge \\
 & \SysNeverStuckP{I}{(\Sys{(D\wedge\Next{\smallangle})}{\W_{QC}}{\Delta})}
\end{array}
\]
where $\W_{QC} \rightarrow \W$ states that $\W$ is an abstraction of
$\W_{QC}$.

\subsection{Reuse via Substitution}
\label{sec:simulation}
\label{sec:subst}
Substitution of expressions for variables is a simple but powerful operator
that allows us to reuse controllers and their properties.  For example,
substitution allows us to perform familiar geometric transformations such
as translations, reflections, scaling, and rotations.  In addition,
substitution allows us to project simple dynamics onto more complex
dynamics; a technique we use to build verified state-constraining
controllers for the quadcopter.  We will explain the general technique by
using it to transport (re-use) the second-derivative controller
(\derivShimx), and its safety proof, to enforce a maximum altitude for our
quadcopter.

For a formula $P$ and substitution $\sigma$ (map from variables to
expressions), the semantic definition of substitution is:
\[\begin{array}{rcl}
tr \models \Rename{\sigma}{P} & \triangleq & \Rename{\sigma}{tr} \models P
\end{array}
\]
which states that a substituted formula ($\Rename{\sigma}{P}$) holds on a
trace ($tr$) if the formula ($P$) holds on the renamed trace
($\Rename{\sigma}{tr}$).  Under this definition, application of
substitution always guarantees preservation:
\begin{theorem}{\ProofRule{SubstPreserves}}
\begin{prooftree}
\AxiomC{$\entails \SysPreservesP{I}{S}$}
\UnaryInfC{$\entails \SysPreservesP{\big(\Rename{\sigma}{I}\big)}{\big(\Rename{\sigma}{S}\big)}$}
\end{prooftree}
\end{theorem}
This proof rule allows us to easily
transport \ProofRule{$\partial^2$-Preserves} to the quadcopter.  For
example, using it we can conclude
\begin{align*}
&\entails \SysPreservesP{\big(\Rename{\sigma_{\partial^2\rightarrow{}QC}}{\invShimx}\big)}{\big(\Rename{\sigma_{\partial^2\rightarrow{}QC}}{\derivShimx}\big)} \\
\sigma_{\partial^2\rightarrow{}QC} &\defined \{\tlavar{a} \mapsto \tlavar{T} \cos \tlavar{\roll} \cos \tlavar{\pitch} - g ,\; \tlavar{y} \mapsto \tlavar{z} ,\; \tlavar{v} \mapsto \tlavar{v_z}\}
\end{align*}
Note that the first argument to $\SysPreserves{}$ is the inductive
invariant for the new system, and can be read directly from the conclusion
of the preservation theorem.  This is the case for all of our preservation
theorems.

Next, we need to justify the \progress{} of the substituted system.  The
interaction between substitution and \progress{} is a bit subtle because
substitutions can introduce coupling between values that were uncoupled
before the substitution.  For example, $(\tlanextvar{x} =
1 \wedge \tlanextvar{y} = 0)$ is \Enabled{} while
$\Rename{\tlavar{x} \mapsto \tlavar{z}
,\; \tlavar{y} \mapsto \tlavar{z}}{(\tlanextvar{x} =
1 \wedge \tlanextvar{y} = 0)}$, which equals $\tlanextvar{z} =
1 \wedge \tlanextvar{z} = 0$, is not.

However, we can prove that \emph{invertible} substitutions preserve
progress.  This is because the inverse of the substitution is a function
for computing the \Enabledness witness for the substituted system from
the \Enabledness witness for the original system.  Because of this, we can
actually state a stronger \progress{} theorem for substitution, which
captures the fact that the inverse substitution preserves known constraints
on \Enabledness witnesses, such as $\Ca$ for the first and second
derivative controllers.  As we will see, this is crucial for proving the
small-angle constraint ($\smallangle$).  Formally,
\begin{theorem}{\ProofRule{Subst\xmakefirstuc{\progress{}}}}
For all formulas $S$, state formulas $Q$ and $R$, and substitutions
$\sigma$, if there exists a $\sigma^{-1}$ such that
$R\entails(\sigma \circ \sigma^{-1})\, \tlavar{x} = \tlavar{x}$ for all
variables \tlavar{x} that occur primed in $S$, and if
$R \entails \Rename{\sigma^{-1}}{Q}$ then
\begin{prooftree}
\AxiomC{$\entails \SysNeverStuckP{I}{(S\wedge\Next{R})}$}
\UnaryInfC{$\entails \SysNeverStuckP{\big(\Rename{\sigma}{I}\big)}{\big(\Rename{\sigma}{S}\wedge\Next{Q}\big)}$}
\end{prooftree}
\end{theorem}

When we apply this theorem to prove the \progress{} of the quadcopter
altitude controller, the following inverse works:
\[
\sigma_{\partial^2\rightarrow{}QC}^{-1} \triangleq \tlavar{\roll} \mapsto 0, \tlavar{\pitch} \mapsto 0, \tlavar{T} \mapsto \tlavar{a} + g, \tlavar{z} \mapsto \tlavar{y}, \tlavar{v_z} \mapsto \tlavar{v}
\]
We instantiate $Q$ with $\smallangle$ and $R$ with $\Ca$ to guarantee
\[
\SysNeverStuckP{I}{\big(\Sys{(\Rename{\sigma_{\partial^2\rightarrow{}QC}}{D_{\derivShimx}}\wedge \Next{\smallangle})}{(\Rename{\sigma_{\partial^2\rightarrow{}QC}}{\W_{1D}})}{\Delta}\big)}
\]

Finally, as noted above, we need to prove that
$\W_{QC} \rightarrow \Rename{\sigma_{\partial^2\rightarrow{}QC}}{\W_{1D}}$,
i.e. the continuous dynamics produced by the substitution is an abstraction
of the full quadcopter dynamics.  This reasoning involves standard
substation within differential equations, which we have formalized and
proved sound in Coq, and mechanical arithmetic reasoning.

\paragraph*{Enforcing Planar Boundaries}
Using our two substitution theorems, we can map the first- and
second-derivative controllers onto the quadcopter dynamics in many ways,
allowing us to verify many properties with relatively little effort.  We
showed how to use it to implement an upper bound on altitude.  In general,
we can use substitution on the second derivative controller to enforce that
the quadcopter stays on one side of any 2D plane in 3D space.  For example,
we can enforce a maximum-west boundary to prevent the quadcopter from
flying into a building.  Similarly, by applying substitution to the
first-derivative controller we can place bounds on velocity in any
direction.  The key is that our two substitution theorems allow us to
transfer the correctness of any controller to work on a new dynamics that
can be constructed from an invertible substitution.

\subsection{Disjunctive Composition}
\label{sec:disjunctive-composition}

In this section we present rules to compose systems using disjunction.  For
example, suppose that we wish to enforce a rectangular no-fly zone centered
around the origin (depicted in Figure~\ref{fig:dont-hit-the-pilot}).  A
system can avoid the no-fly zone if at all times it is to the north, the
south, the east, \emph{or} to the west of the rectangle.  We can build such
a system by disjoining subsystems $M_N$, $M_S$, $M_E$, and $M_W$, which are
each built from a substitution applied to \derivShimx{} to enforce the
northern, southern, eastern, and western boundaries of the box,
respectively.  As we will see, separately exposing preservation
and \progress{} allows us to define a disjunction operator that is fully
compositional and that permits the system to transition from an inductive
invariant of one subsystem to another (e.g. north of the no-fly zone to
west of the no-fly zone) during a single trace; this would not be possible
with global invariant proofs.

The disjunctive composition of two systems is defined by the $\oplus$
operator, which is indexed by the inductive invariants of the two systems.
Formally,
\begin{definition}{Disjunctive composition}
\[\begin{array}{l}
\SysDisjoin{\left(\Sys{D_1}{\W}{\Delta}\right)}{I_1}{\left(\Sys{D_2}{\W}{\Delta}\right)}{I_2} \triangleq \\
\qquad \Sys{\big((I_1 \wedge D_1) \vee (I_2 \wedge D_2)\big)}{\W}{\Delta} %% \big((I_1 \wedge \W_1) \vee (I_2 \vee \W_2)\big)}{\Delta}
\end{array}
\]
\end{definition}
The inclusion of the \emph{inductive} invariants in this definition is
essential to enforce the disjunction of the properties.  To see why,
consider our example and suppose the system is currently along the western
edge of the no-fly zone.  Since the system is outside of the inductive
invariant of the eastern controller, the eastern controller could allow the
system to do anything, including moving east into the no-fly zone.  Thus,
at each discrete transition, the composed controller must consider the
discrete transitions of a sub-controller whose inductive invariant
currently holds.  This guarantees that the inductive invariant of that
subsystem holds after the transition.  Moreover, it means that the system
can transition from one inductive invariant to another, only where the
inductive invariants overlap.

\begin{figure}[t]
\centering
\begin{tikzpicture}[x=1cm,y=1cm]

\begin{scope}[yscale=0.75,xscale=0.75]

% Axis
\draw (-3,0) -- (3,0) ;
\draw (0,-3) -- (0,3) ;


\draw[dashed] (-1.75,-3) -- (-1.75,3) ;
\fill[pattern=north west lines, pattern color=gray] (-1.75,-3) rectangle (-3,3) ;

\begin{scope}[rotate=180]
\draw[dashed] (-1.75,-3) -- (-1.75,3) ;
\fill[pattern=north west lines, pattern color=gray] (-1.75,-3) rectangle (-3,3) ;
\end{scope}

\begin{scope}[rotate=90]
\draw[dashed] (-1.5,-3) -- (-1.5,3) ;
\fill[pattern=north east lines, pattern color=gray] (-1.5,-3) rectangle (-3,3) ;
\end{scope}

\begin{scope}[rotate=-90]
\draw[dashed] (-1.5,-3) -- (-1.5,3) ;
\fill[pattern=north east lines, pattern color=gray] (-1.5,-3) rectangle (-3,3) ;
\end{scope}

\node[anchor=west,fill=white,opacity=0.75,inner sep=0.15em] at (-2.7,2.5) {\small Safe Region} ;
\node[anchor=west,inner sep=0.15em] at (-2.7,2.5) {\small Safe Region} ;

\node[anchor=west,fill=white,opacity=0.75,inner sep=0em,text width=1.25cm,align=center] at (-1.75,0.5) {} ;
\node[anchor=west,inner sep=0em,text width=2cm,align=center] at (-1.5,0.1) {\small Restricted Zone} ;

\draw[latex-] (2,2) to[bend left] (3.5,2) node[inner sep=0.15em,anchor=north west,text width=3.5cm] {\footnotesize Intersection of inductive invariants} ;

\draw[latex-] (-2,0.1) to[bend right] (-3.5,0.4) node[inner sep=0.15em,anchor=east] {\footnotesize $M_W$} ;
\draw[latex-] (-1,1.75) to[bend left] (-3.5,2.1) node[inner sep=0.15em,anchor=south east] {\footnotesize $M_N$} ;
\draw[latex-] (-1,-1.75) to[bend left] (-3.5,-2.1) node[inner sep=0.15em,anchor=south east] {\footnotesize $M_S$} ;

\draw[latex-] (2,-0.5) to[bend right] (3.5,-1.2) node[inner sep=0.15em,anchor=west] {\footnotesize $M_E$} ;

\end{scope}
\end{tikzpicture}


\caption{Staying out of restricted airspace using the disjunction of four controllers.}
\label{fig:dont-hit-the-pilot}
\end{figure}

This definition of disjunction is fully compositional with
both \SysPreserves and \SysNeverStuck.  In particular, the disjunctive
composition of two systems that independently preserve $I_A$ and $I_B$
preserves $I_A \vee I_B$:
\begin{theorem}{\ProofRule{DisjoinPreserves}}
\[\begin{array}{rl}
       & \SysPreservesP{I_A}{A} \;\wedge\; \SysPreservesP{I_B}{B} \\
\entails & \SysPreservesP{\big(I_A \vee I_B\big)}{\big(\SysDisjoin{A}{I_A}{B}{I_B}\big)}
\end{array}
\]
\end{theorem}
\noindent
Using only this theorem we can easily construct a proof that the
disjunctive composition of our four controllers enforces the no-fly zone
property.

A similar theorem states that \disjoinop guarantees \SysNeverStuck.
\begin{theorem}{\ProofRule{Disjoin\xmakefirstuc{\progress{}}}}
\[\begin{array}{rl}
       & \SysNeverStuckP{I_A}{A} \;\wedge\; \SysNeverStuckP{I_B}{B} \\
\entails & \SysNeverStuckP{(I_A \vee I_B)}{\big(\SysDisjoin{A}{I_A}{B}{I_B}\big)}
\end{array}
\]
\label{thm:progress-disjoint}
\end{theorem}
\noindent
The proof follows from the fact that $I_A$ ensures the \Enabledness of $A$
and $I_B$ ensures the \Enabledness of $B$.  Since the new inductive
invariant $\left( I_A \vee I_B\right)$ ensures that at least one of $I_A$
or $I_B$ holds, at least one of $A$ or $B$ must be \Enabled.

Disjunction is a very powerful composition mechanism that is applicable in
a wide variety of circumstances.  For example, we can use it to guarantee
that a train can only have a high velocity when it is not in a curve by
composing a maximum velocity controller with a controller that stops the
train before curves.  A controller such as this one could have prevented
the Amtrak derailment in Philadelphia in 2015 that killed 8 people and
injured 200.

\subsection{Conjunctive Composition}
\label{sec:conjunctive-composition}

In this section, we present rules to compose systems to ensure the
conjunction of their properties.  For example, one might want to ensure
that a system's upward velocity does not exceed 1 m/s \emph{and} that the
system stays below 100m by conjoining the first and second derivative
controllers.  Unlike disjunctive composition, conjunctive composition of
two systems satisfying \progress{} does not guarantee a system
satisfying \progress{}, due to coupling.  However, our ability to
separately prove \progress{} allows us to push forward.  We show that, even
in coupled domains, conjunctive composition can lead to substantial savings
in proof effort, and demonstrate some clever tricks that allow us to
decouple domains that, on the surface, seem intricately linked.

We use the same definition for conjoining systems as in Chapter~\ref{chap:memo15}.
We restate the definition here for clarity:
\sysconjoin*
The crucial feature of this definition is how it interacts
with \SysPreserves.  Note that while disjunctive composition requires that
both systems have the same continuous transition, conjunctive composition
does not impose this restriction.
\begin{theorem}{\ProofRule{ConjoinPreserves}}
\[\begin{array}{rl}
       & (I_B \rightarrow \SysPreservesP{I_A}{A}) \;\wedge\; (I_A \rightarrow \SysPreservesP{I_B}{B}) \\
\entails & \SysPreservesP{\big(I_A \wedge I_B\big)}{\big(\SysConjoinP{A}{B}\big)}
\end{array}
\]
\label{thm:conjoin-preserves}
\end{theorem}
Intuitively, if we start in a state satisfying both $I_A$ and $I_B$, $A$
guarantees that we stay in $I_A$, and $B$ guarantees that we remain in
$I_B$, then if both $A$ and $B$ hold, we must remain in the intersection of
$I_A$ and $I_B$. Note that Theorem~\ref{thm:conjoin-preserves} removes the
restriction of acyclic parallel composition from
Theorem~\ref{thm:sys-compose} in Chapter~\ref{chap:memo15}. This is
possible because the inductive invariants are made explicit.

The difficulty of conjunctive composition lies in justifying \progress{}.
Even though $A$ and $B$ may independently be \Enabled under the inductive
invariant, there is no guarantee that their conjunction is \Enabled.  For
example, suppose that we wish to compose an overly-conservative upper-bound
controller that insists on a negative acceleration and a similarly
conservative lower-bound controller that insists on a positive
acceleration.  Since acceleration can not be simultaneously positive and
negative, the conjunction of these controllers does not
satisfy \progress{}.

Nevertheless, all is not lost when conjoining two systems.  There are a
variety of techniques for proving \Enabledness of conjunctions.  We will
illustrate these techniques through a sequence of examples, ultimately
culminating in a 3D bounding box for both position and velocity.

\paragraph*{Example: Staying within an Interval}
Consider constructing a controller that enforces both an upper and a lower
bound on both position ($y$) and velocity ($v$) in a single spatial
dimension.  We can build such a controller (which we call \intervalShim)
using $\SysConjoin$ and an application of our substitution operator to the
second- and first-derivative controllers:
\begin{align*}
\intervalShim &\defined \SysConjoinP{\derivShimx}{\Rename{\mirrorx}{\derivShimx}} \SysConjoin \derivShimv \SysConjoin \Rename{\mirrorx}{\derivShimv} \\
\mirrorx &\defined \{ \tlavar{y} \mapsto -\tlavar{y} ,\; \tlavar{v} \mapsto -\tlavar{v} ,\; \tlavar{a} \mapsto -\tlavar{a} \}
\end{align*}
Here, the $\mirrorx$ substitution mirrors the controller's logic so that
rather than enforcing an upper bound on $\tlavar{y}$ of $\ubY$ (resp. $v_y$
of $v_{\mathsf{ub}}$), the substituted controller enforces a lower bound on
$\tlavar{y}$ of $-\ubY$ (resp. $v_y$ of $-v_{\mathsf{ub}}$).

The preservation of this composition follows immediately
from \ProofRule{ConjoinPreserves}, \ProofRule{$\partial^2$-Preserves}, \ProofRule{$\partial$-Preserves},
and \ProofRule{SubstPreserves}.  However, since conjoined systems are not
guaranteed to satisfy \progress{}, we must prove this separately.
Formally, we must prove \progress{} of $\intervalShim$ under the
conjunction of the inductive invariants of the subsystems:
\footnote{For brevity, $\invShimx$ and $\invShimv$ refer to the inductive invariants of \derivShimx{} and \derivShimv{}.}
\[
\SysNeverStuckP{\big(\invShimx\wedge\Rename{\mirrorx}{\invShimx}\wedge\invShimv\wedge\Rename{\mirrorx}{\invShimv}\big)}{\intervalShim}
\]

Informally, this states that, under the inductive invariant, there exists
an action that is acceptable to all of the (non-deterministic) controllers.
More formally, we must justify that there does not exist any state
satisfying the inductive invariant for which there is not an action that
all controllers accept.  Unfolding the definitions and applying the
substitution ($\mirrorx$) reveals that the \progress{} of $\intervalShim$
reduces to first-order reasoning over real arithmetic.  At first glance, it
may seem as though modularity was a failure here; however, by separating
preservation and \progress{}, the non-modularity of \progress{} did not
prevent us from modularly proving preservation.  Moreover, our split
crucially allows us to assume the inductive invariant when
proving \progress{}.

This is in contrast to other work~\cite{alur1997modularity} where
conjunctive (parallel) composition is only allowed when the individual
modules output to disjoint sets of variables.  Furthermore, we found the
proof of $\SysNeverStuckP{I_{\intervalShim}}{\intervalShim}$ to
be \emph{less than one third} the size of the proof of preservation of the
individual second-derivative controller \derivShimx.  This means that
$\SysConjoin$ greatly reduces the cost of conjoining systems, even when
these systems are tightly coupled.

It is important to note that under-specification of controllers is
essential to composition in this case.  If both systems were completely
specified then the two systems would be forced to make identical choices in
order for them to compose.  By composing under-specified controllers, we
retain the flexibility to find a solution that reconciles the restrictions
of both controllers.

In certain cases, our proof rules can be used to verify \progress{}
compositionally.  To demonstrate this, we turn to the task of using our
interval controller to modularly build a bounding rectangular prism.  We
approach the problem in three steps.  First, we compose two interval
controllers to enforce a bounding box in two spatial dimensions.  In the
next section, we adapt this controller to the quadcopter by incorporating
the small-angle constraint.  Finally, we apply the same technique to extend
the 2D box into a 3D prism.

\paragraph*{Example: Conjunction of Independent Systems}
We construct the box by conjoining two instances of $\intervalShim$, using
substitution to map them to the $x$- and $z$-dimensions respectively.
\[\begin{array}{rcl}
\mathsf{Box} & \defined & \Rename{\sigma_x}{\intervalShim} \SysConjoin \Rename{\sigma_z}{\intervalShim} \\
\sigma_x &\defined &\{ \tlavar{y} \mapsto \tlavar{x} ,\; \tlavar{v} \mapsto \tlavar{v_x} ,\; \tlavar{a} \mapsto \tlavar{a_x} \} \\
\sigma_z &\defined & \{ \tlavar{y} \mapsto \tlavar{z} ,\; \tlavar{v} \mapsto \tlavar{v_z} ,\; \tlavar{a} \mapsto \tlavar{a_z} - g \}
\end{array}
\]
Verifying that $\mathsf{Box}$ enforces a bounding box is simply a matter of
using \ProofRule{ConjoinPreserves}, \ProofRule{SubstPreserves}, and the
preservation proof of $\intervalShim$.

As others have noted~\cite{alur1997modularity}, \progress{} of conjoined
systems is compositional, if the two systems output to disjoint sets of
variables.  In our logic, the output variables of a formula are the
next-state (primed) variables, whose disjointness is expressed by $A \perp'
B$.  Formally,
\begin{theorem}{\ProofRule{Conjoin\xmakefirstuc{\progress{}}Disjoint}}
For all systems $A$ and $B$ such that $A \perp' B$, and for all state formulas $I_A$ and $I_B$,
\[\begin{array}{rl}
       & \SysNeverStuckP{I_A}{A} \;\wedge\; \SysNeverStuckP{I_B}{B} \\
\entails & \SysNeverStuckP{\big(I_A \wedge I_B\big)}{\big(\SysConjoinP{A}{B}\big)}
\end{array}
\]
\end{theorem}
While we can not express disjointness of primed variables ($A \perp' B$)
formally within our temporal logic, we can capture it in Coq's logic.
Further, in practice the disjointness of two formulas is easily decidable
in many cases meaning that discharging this side condition is often
trivial.  From this theorem and variable disjointness, $\mathsf{Box}$
satisfies \progress{} for the dynamics with independent \tlavar{a_x}
and \tlavar{a_y}.

Now suppose that instead of rectangular dynamics with independent control
inputs \tlavar{a_x} and \tlavar{a_y}, we want a controller for polar
coordinates with independent control inputs \tlavar{a} and \tlavar{\pitch}.
For example, these are the dynamics of a 2D version of the quadcopter (with
$\tlavar{\roll}$ fixed at 0), in the absence of the small-angle constraint.
While the transformation to polar coordinates seems to couple the $x$ and
$z$ instantiations of $\intervalShim$, there is always an invertible map
from rectangular to polar coordinates.  This connection between polar and
rectangular coordinates allows us to use the substitution theorems from
Section~\ref{sec:subst} to prove both preservation and \progress{} for a
version of $\mathsf{Box}$ that operates using polar thrust.

Note that to arrive at this result, we first used disjoint composition to
justify the syntactically independent system satisfies \progress{}.  We
then applied our substitution theorems and the invertibility of the
rectangular-to-polar transformation to show that the resulting system was
the same as the quadcopter dynamics modulo the small-angle constraint. The
takeaway is that although the system
($\Rename{\sigma_{\pitch}}{\mathsf{Box}}$) is not superficially composed of
disjoint subsystems, we can still verify both \progress{} and preservation
fully modularly.

\paragraph*{Example: Incorporating the Small-angle Constraint}
The previous example ignored the small-angle constraint, which is critical
to the accuracy of our model of the quadcopter's dynamics.  In this section
we address this shortcoming and complete the verification of $\mathsf{Box}$
with respect to a 2D version of $\W_{QC}$ (again, with $\tlavar{\roll}$
fixed at 0) \emph{in a fully modular way}.  Figure~\ref{fig:small-angle}
shows how the small-angle constraint introduces coupling into the
independent box.  In particular, the hatched region corresponds to the
accelerations achievable under the small-angle constraint while the
unconstrained region corresponds to the accelerations necessary for
$\mathsf{Box}$ developed in the previous section.

\newcommand{\Tminx}{\ensuremath{\Tmin^{x}}\xspace}
\newcommand{\Tminz}{\ensuremath{\Tmin^{z}}\xspace}

\begin{figure}[t]
\centering
\begin{tikzpicture}[x=0.48cm,y=0.48cm]

\draw[latex-latex] (0,-2) -- (0,2) node[above] {$z$};
\draw[latex-latex] (-2,0) -- (2,0) node[anchor=west] {$x$};

\draw[-latex] (0,-1.5) -- (45:3) ;
\draw[-latex] (0,-1.5) -- (135:3) ;
\fill[pattern color=gray!50,pattern=north west lines] (0,-1.5) -- (45:3) -- ++(-4.25,0) -- cycle ;

\draw[densely dotted,thick] (-1.5,-1.5) rectangle (1.5,1.5) ;

\draw[dashed] (-0.6,-0.5) rectangle (0.6,0.5) ;

\begin{scope}[xscale=-1]

\draw[latex-,gray] (1.7,1.75) to[bend left] (3,1) node[anchor=north east,inner sep=0.2,black] {\footnotesize small-angle achievable} ;
\draw[latex-,gray] (1,-1) to[bend left] (3,-1) node[anchor=north east, inner sep=0.2,black] {\footnotesize unconstrained \Tminx, \Tminz} ;
\draw[latex-,gray] (-0.2,0.2) to[bend right] (-3,1.0) node[anchor=west,inner sep=0.2,black,text width=1.8cm] {\footnotesize independently achievable} ;

\end{scope}
\end{tikzpicture}

\caption{Decoupling of \tlavar{a_x} and \tlavar{a_z}.}
%%\caption{\tlavar{a_x} and \tlavar{a_z} can be decoupled under the small-angle constraint under assumptions on their ranges.}
\label{fig:small-angle}
\end{figure}

Our approach relies on a detail of the verification which we glossed over
in the previous presentation.  In particular, because the second-derivative
controller is \emph{parameterized by \Tmin}, $\mathsf{Int}$ is
parameterized by an \Tmin and $\mathsf{Box}$ is parameterized by a pair
of \Tmin's in the $x$ and $z$ dimensions.  The right side of
Figure~\ref{fig:small-angle} shows how we leverage the parameterization of
$\mathsf{Box}$ to logically decouple the two dependent dimensions.  Our
insight is the following.  If we can pick values for \Tminx and \Tminz
(note that \Tminx and \Tminz are negative) such that all accelerations
$a_x$ and $a_z$ where $|a_x| \leq -\Tminx$ and $|a_z| \leq -\Tminz$ are
achievable by \tlavar{T} and \tlavar{\pitch} under the small-angle
constraint then the values of \tlavar{a_x} and \tlavar{a_z} can be chosen
independently \emph{if they fall within the bounds}.  Purely trigonometric
reasoning reveals that the constraint on \tlavar{\pitch} is achieved for
any values of \tlavar{a_x} and \tlavar{a_z} satisfying
$\xConstraint \wedge \zConstraint$ defined as:
\begin{align*}
\xConstraint &\defined -(\Tmin + g) \tan 30\mydegree \leq \tlavar{a_x} \leq  (\Tmin + g) \tan 30\mydegree \\
\zConstraint &\defined \Tmin \leq  \tlavar{a_z}  \leq   -\Tmin
\end{align*}

We can formalize this reasoning by noting that the previous example
($\mathsf{Box}$), instantiated with the above choices of $\Tminx$ and
$\Tminz$, satisfies \progress{} under the constraint
$\xConstraint \wedge \zConstraint$.  Thus, we can use the substitution
proof rule \ProofRule{Subst\xmakefirstuc{\progress{}}} with a
rectangular-to-polar substitution, instantiating $R$ with
$(\xConstraint\wedge\zConstraint)$ and $Q$ with the small-angle condition
(ignoring $\tlavar{\roll}$ since it is assumed to be 0 in this 2D example).

The important point of this verification is that the separation of
preservation and \progress{} allowed for a relatively small and modular
proof of a complex property of a \emph{highly coupled system}.  Without the
separation of preservation and \progress{}, the coupling would have forced
us to re-prove many of the intermediate properties.

\paragraph{Example: Adding the Third Dimension}
We can build a controller enforcing a cube by conjoining $\mathsf{Box}$
with another instance of $\intervalShim$ substituted to reflect the $y$
dimension.  We can then apply this cube to the 3D version of $\W_{QC}$
(including $\tlavar{\roll}$ and the coupling small-angle constraint).
Again, the modularity of our proofs shields us from much of the complexity.
$\mathsf{Box}$ enforced the constraints for \tlavar{x} and \tlavar{z}
using \tlavar{a_x} and \tlavar{a_z}.  Extending this to handle the third
dimension simply requires that we use substitution to view \tlavar{a_x}
and \tlavar{a_z} as a single unit and carry out the same reasoning,
independently controlling that composed unit and \tlavar{a_y}.

\section{Evaluation}
\label{sec:eval}
We evaluate our approach with four criteria: (1) proof effort, (2)
expressiveness, (3) flight behavior, and (4) comparison to fully automated
tools.

\subsection{Proof Effort}
Table~\ref{fig:monitors} lists all of the geofencing controllers that we
verified and flew, along with the composition mechanism, proof size, and
number of symbols in the discrete transition for each one.  The first 2
rows list controllers that we ported from prior work; these are our atomic
building blocks.  The next 3 rows list controllers that we built and
verified using a combination of substitution and conjunction.  The
remaining rows list controllers that we built using disjunction and
substitution.

\begin{table}
\caption{Systems implemented and proved correct. The first column is the name of the system. The second column shows how each system was built and proved correct from smaller components. $\Rename{}{i}$ indicates a substitution applied to system $i$ (we have omitted the specific substitution); $\SysConjoin$ represents conjunction composition; and $\disjoinop$ represents disjunction composition. The third and fourth columns show the number of symbols in the discrete controller and the number of lines of proofs, respectively.}
\label{fig:monitors}
\begin{tabular}{l|l|r|r}
id: Name & Built How? & \# of Symbols & Lines of Proof \\
\hline 
a: 1D velocity bound & From Scratch & 43 & 130 \\
b: 1D position bound & From Scratch & 126 & 484 \\
c: 1D interval & $\SysConjoinP{\SysConjoinP{\Rename{}{b}}{\Rename{}{b}}}{\SysConjoinP{\Rename{}{a}}{\Rename{}{a}}}$ & 323 & 194 \\
d: 2D square & $\Rename{}{(\SysConjoinP{\Rename{}{c}}{\Rename{}{c}})}$ & 805 & 258 \\
e: 3D cube & $\Rename{}{(\SysConjoinP{d}{\Rename{}{c}})}$ & 1353 & 201 \\
f: 3D square donut & $\Rename{}{e}\disjoinop\Rename{}{e}\disjoinop\Rename{}{e}\disjoinop\Rename{}{e}$ & 5412 & 23 \\
g: 3D +, T, $\bot$ & $\Rename{}{e}\disjoinop\Rename{}{e}\disjoinop\Rename{}{e}\disjoinop\Rename{}{e}$ & 5412 & 23 \\
h: 3D pilot box & $\Rename{}{e}\disjoinop\Rename{}{e}\disjoinop\Rename{}{e}\disjoinop\Rename{}{e}$ & 5412 & 23 \\
\end{tabular}
\end{table}

Note that our 1D position controller ($b$ in Table~\ref{fig:monitors})
involves 3 variables and 126 symbols and is a relatively simple discrete
transition requiring only multiplication and addition.  However, verifying
this position controller required substantial, tedious, manual proof effort
because of the difficulty of reasoning about nonlinear real arithmetic.  On
the other hand, our 3D cube controller ($e$ in Table~\ref{fig:monitors})
contains 9 variables, 1353 symbols, trigonometric functions, and angular
constraints.  This increase in complexity is actually quite daunting for
two reasons: (1) arithmetic decision procedures have very bad worst case
complexity, and (2) trigonometric functions make the problem
undecidable~\cite{HarrisonLogic09}.  Despite the substantial increase in
complexity, our modular verification approach allowed us to verify the 3D
version with only a modest amount of additional manual proof effort.

Moreover, note that all controllers built using disjunction require
negligible proof effort on top of the effort required to verify the
individual components.  This is because disjunction composes proofs of both
preservation and \progress{}. As we discuss in the next subsection, this is
a very powerful tool.

Finally, it is worth noting that we have several admits in our Coq
development: (1) basic arithmetic theorems, (2) some theorems bridging the
gap between $\W_{QC}$ and its abstractions, (3) theorems
stating \progress{} of the continuous transitions in the quadcopter model.
Crucially, these admits do not interfere with the core ideas that we are
exploring, namely modular reasoning about sampled-data systems.

\subsection{Expressiveness}
Disjunction is extremely powerful when building real-world controllers.
For example, we can build up pixelated versions of arbitrary shapes by
composing instances of our cube controller.  Interesting properties that we
can build with this include: (1) avoid no-fly zones such as airports, the
white house, etc, (2) do not fly within a box surrounding the pilot, (3)
stay away from static barriers such as trees and buildings, and (4) avoid
skyscrapers in a city. In fact, our disjunction rules can be used to
compose \emph{any} controller with any other controller, not just a cube.
For example, if we were able to verify a controller enforcing a triangular
prism or a sphere, we could compose instantiations of these controllers
with our existing cube, with each other, or with any other controller.
Again, all of this requires negligible additional proof burden.

Conjunctive composition is similarly expressive -- the box and the cube
both make heavy use of conjunctive composition.  The proof burden when
using conjunction comes from justifying \progress{} (progress) of the
resulting systems, which is not as compositional, especially in highly
coupled domains.  Nevertheless, Our separation of \progress{} (progress)
and preservation makes it possible to build compositional proofs of
preservation even in highly coupled domains such as the box and cube.
Finally, substitution allows us to reuse all controllers by translating and
rotating them, and by transforming them into controllers for different
physical dynamics.

\subsection{Behavior in Actual Flight}
\label{sec:flight}
To make sure that our controllers work well in practice, we manually
translated our models to C and ran them on a 3DR Iris+ quadcopter.  This
allows us to evaluate the accuracy of our approximations.
Figure~\ref{fig:emsoft-arch} depicts the architecture of the system with one of
our controllers inserted.  Our controllers check the outputs of the
pre-existing control software, potentially replacing them with default safe
values; this resolves the non-determinism of the controller specifications.
This architecture means that the behavior of much of the existing control
software (left of ``Verified Control'') cannot cause a violation of the
fence.  Moreover, if the quadcopter remains sufficiently far from the fence
boundaries, the system behaves exactly as it would without our controllers,
so any performance properties of the existing control software are
preserved.

\begin{figure}[t]
  \centering
  \begin{tikzpicture}
    \tikzstyle{every node}=[draw=black,thick,font=\bfseries,node distance=0.35cm]
    \tikzstyle{every path}=[draw=black,thick]
    \tikzstyle{wnode}=[minimum height=1cm]

    \node[fill=black,text=white,text width=2cm,align=center,minimum height=1cm] (mon) at (0,0) {Verified Control} ;
    \node[wnode,left=of mon,text width=2cm,align=center] (existing)  {Existing Control Software} ;

    \node[draw=black,left=of existing,minimum height=0.5cm,yshift=0.3cm] (user) {User} ;
    \node[draw=black,left=of existing,minimum height=0.6cm,yshift=-0.8cm] (sensor) {Sensors} ;

    \node[wnode,right=of mon,text width=2cm,align=center] (control)  {Attitude Control} ;
    \node[wnode,right=of control] (motor)  {Motors} ;

    \newcommand{\quadline}[2]{%
    \foreach \i in {-2,...,1}{%
      \draw[-latex] ([yshift=0.12cm + \i * 0.24 cm]#1.east) -- ([yshift=0.12cm + \i * 0.24 cm]#2.west) ;}}

    \quadline{mon}{control}
    \quadline{control}{motor}
    \quadline{existing}{mon}
    \draw[-latex] let \p1 = (user.east),
                      \p2 = (existing.west) in
                  (user.east) -- (\x2,\y1) ;
    \draw[-latex] let \p1 = (sensor.east),
                      \p2 = (existing.west) in
                  ([yshift=0.2cm]sensor.east) -- (\x2,\y1+0.2cm) ;
    \draw[-latex] ([yshift=-0.2cm]sensor.east) -| (mon.south) ;



  \end{tikzpicture}

     \caption{System architecture.}
     \label{fig:emsoft-arch}
\end{figure}

We flew our quadcopter with all controllers listed in
Table~\ref{fig:monitors}.  All of the controllers enforced their
respective bounds with only minor violations, which can be attributed to
un-modeled forces such as wind and to modeling approximations.

Perhaps the largest gap between our temporal logic model and the real world
comes from our assumption that the controllers can instantaneously set the
orientation, or attitude, of the quadcopter
(Section~\ref{sec:quadcopter-dynamics}).  This is not possible in reality.
Instead, an attitude controller controls attitude indirectly by sending
voltage signals to the quadcopter's motors.  Thus, our controllers output
desired attitude values (i.e. roll and pitch) to an attitude controller,
and assume that these values are achieved instantaneously.  Prior work has
suggested that this is a reasonable approximation, under the small angle
assumption, since the attitude dynamics are significantly faster than the
velocity and position dynamics~\cite{Gillula2011}.  Nevertheless, one of
the reasons for running our controllers on a real quadcopter is to
experimentally evaluate whether the attitude assumption works well in our
context.

Manually translating our models to C introduces a potential source of
errors.  The two main aspects of this translation are: (1) approximating
infinite precision real numbers in the logic using floating point numbers,
and (2) validating the assumptions about the runtime of code using
worst-case execution time analysis on the resulting code.  Both of these
problems have been explored in a variety of
contexts~\cite{darulova2014sound,maroneze2014certified,wilhelm2008worst},
and are orthogonal to modularity.

Our experiences flying the controllers from this chapter suggest several
interesting avenues for future study.  First, our controllers can cause the
quadcopter to oscillate near the boundaries.  If the pilot commands maximum
acceleration towards a boundary, then our controllers, rather than
converging smoothly to zero acceleration at the boundary, ultimately
oscillate quickly between the pilots desired acceleration and the
controller's breaking acceleration.  This is due to a large discontinuity
in control input (desired attitude) near the boundaries. We address this
shortcoming in Chapter~\ref{chap:exp-smpl}.

Second, an early C implementation of our multi-cube controller (formed from
a disjunction of cubes), was overly conservative due to the logic it used
for responding to violations of both controllers.  In particular, it
sometimes prevented the quadcopter from leaving a particular cube, even
though the desired quadcopter path was safe for another cube.  The solution
was to break ties in favor of the least invasive choice, which is a
heuristic not expressed in the model. This experience suggests the some
sort of liveness property is an interesting avenue for future work in this
domain.
 
Finally, our results suggest that minor violations of even formally
verified properties are inevitable since no useful model of the physical
world can capture everything.  Most of the time our controllers work well
outside the boundaries, but our experiments do show that occasionally they
cause the quadcopter to temporarily get stuck outside the boundaries. This
means that it would be valuable to prove stronger versions of our state
invariants expressing a form of robustness. Here, we can build on
robustness notions from control theory called input-to-state
stability~\cite{ISSsontag2008} and input-to-state dynamical
stability~\cite{gruneISDS02}.  These definitions state that the behavior of
a system \emph{degrades gracefully} as disturbances grow and that the
effect of individual disturbances decreases as time passes.  For example, a
quadcopter running our controllers may violate the boundary by a small
amount if there is a small amount of wind and by a large amount if there is
a strong wind, but in the absence of wind the vehicle must re-enter the
boundary.

\subsection{Comparison with fully automated tools}
There are a number of state-of-the-art tools that attempt to automatically
verify hybrid
systems~\cite{PHAVerSTTT08,chen2015flow,kong2015dreach,HyTechCAV97}, but
due to the complexity of the domain, they are limited to certain classes of
systems and properties.

PHAVer~\cite{PHAVerSTTT08}, which we ran through the SpaceEx tool
platform~\cite{frehse2011spaceex-small}, is able to verify only one of our
systems, namely the combined upper and lower bound on velocity.  However,
PHAVer is not able to verify the upper bound on velocity in isolation,
probably because there are fewer constraints on acceleration to limit the
search space, compared to the controller that bounds velocity from above
and below.  Finally, PHAVer is not able to run any of our other systems
because the discrete transitions involve non-linear arithmetic; analyzing
these systems would require manual construction of a linear
overapproximation of the discrete transitions.

dReach~\cite{kong2015dreach} and Flow*~\cite{chen2015flow}
are \emph{bounded} model checkers, which means that they can conclude
safety of a system within a user-specified time bound.  Our Coq proofs, on
the other hand, guarantee safety for infinite runs.  It is possible to use
dReach and Flow* to guarantee safety for all runs by manually providing the
inductive invariant.  In this way, our results are complementary, as they
provide a decomposition technique to manage scalability issues of these
tools.  However, these tools are currently unable to handle universally
quantified variables with unbounded domains.  This prevents them from
verifying safety of systems with symbolic parameters, such as $\amin$ --
they require concrete numerical bounds on these parameters, which weakens
the safety theorem.

\section{Acknowledgments}
This chapter, \emsoftack{}


\chapter{Barrier Certificates}
\label{chap:exp-smpl}
In this chapter, we return to a controller that prevents a vehicle from
violating a boundary in one spatial dimension. Controllers with this
objective were described in Chapters~\ref{chap:memo15}
and~\ref{chap:emsoft16}. However, as described in Section~\ref{sec:flight}
of Chapter~\ref{chap:emsoft16}, vehicles running those controllers exhibit
significant oscillation near the boundary due to a discontinuity in the
value output by the controllers. This oscillation, known as chattering in
control theory~\cite{utkin06chatter}, occurs in practice in the presence of
discontinuous control because of a combination of un-modeled dynamics and
sampled-data implementation. The oscillation is undesirable from the
perspective of the pilot. More significantly, it results in a discrepancy
at the boundary between the position dynamics in the model and the position
dynamics in reality, partially causing the boundary violations that we
experienced in actual flight tests.

Our goal was to build a controller whose behavior at the boundaries was
acceptable for the popular open source UAV autopilot called
Ardupilot~\cite{ardupilot}. We worked with Ardupilot developers to build
such a controller as a component of their new geofence module, which
prevents vehicle from exiting a specified safe zone, regardless of what the
pilot does. Crucially the controller that we built did not have a
discontinuity at the boundary, and flight tests confirmed that this
eliminated violations. After building the controller, we attempted to
verify safety of a model of the controller in one spatial dimension.
However, while attempting this task, we found several important gaps in
existing work on formal verification of cyber-physical systems. In this
chapter, we present several proof rules and techniques that address these
shortcomings, and apply them to verify a double integrator model of the
Ardupilot controller in one spatial dimension.

First, deductive techniques for hybrid systems typically involve some
continuous analogue of induction, such as differential
induction~\cite{Platzer10DAL}, whose mechanical verification we
discussed in Chapter~\ref{chap:memo15}, or barrier
certificates~\cite{prajna04barrier}. These techniques provide conditions
for verifying that a state predicate is an invariant of a system of
differential equations, without actually solving the system of
equations. However, differential induction and the original formulation of
barrier certificates from~\cite{prajna04barrier} are too weak to verify
invariants for certain systems, particularly those whose solutions
exponentially decay towards the invariant boundary. For example, these
techniques cannot verify invariance of $y \leq 0$ along solutions of the
system $\dt{y} = -y$. Systems of this form arise naturally when building a
controller like the one designed for Ardupilot -- the controller should
allow the vehicle to approach the boundary and smoothly come to a stop at
the boundary. In order to verify this, differential induction must be
augmented with another proof rule called differential
auxiliaries~\cite{Platzer12diffcut}.

On the other hand, recent work from the control theory
community~\cite{kong2013barrier,xu15barrier,nguyen16barrier} has produced a
new version of barrier certificates that is less conservative than prior
work. In particular, the condition required for invariance captures
exponential decay towards the invariant boundary. We provide the first
implementation of this approach in a formal verification context, and
demonstrate its ease of use on the controller that we designed for
Ardupilot.

Second, control systems are often designed under the assumption that
controllers run continuously, while the actual implementation is a
sampled-data system. In control theory, this process is known as
emulation~\cite{laila02sampled}. System designers can compensate for this
(and other) approximations by adding a safety ``buffer'' to the system. For
example, the Ardupilot controller we helped build stops the vehicle 1 meter
prior to the actual safety boundary. In order to reason about whether such
a buffer is sufficient for a given system, we augmented the new barrier
certificate proof rule
from~\cite{kong2013barrier,xu15barrier,nguyen16barrier} to bound the error
introduced by a continuous time approximation of a sampled-data
system. This rule allows one to perform the majority of reasoning in a
purely continuous model using powerful techniques resulting from over a
century of control theory research. We use this rule to show that the
constant sized buffer is, in fact, sufficient to compensate for the
continuous time approximation of the Ardupilot controller.

Finally, other formal verification frameworks are unable to apply
differential induction and barrier certificates to invariants involving
piecewise functions.  We show how to remove this restriction by working in
the expressive Coq proof assistant. This allows us to verify the critical
piecewise invariant arising in the Ardupilot controller.

\section{Overview}
\label{sec:exp-smpl-overview}
Ardupilot is a popular open source autopilot installed in over 1,000,000
vehicles, including quadcopters and other UAVs~\cite{ardupilot}. The
geofence module allows one to specify a 2D polygon safe region. The pilot
is free to move arbitrarily within the polygon, but the module restricts
movement near the boundary and ultimately stops the vehicle within 1 meter
of the boundary.

We focus on the controller logic for stopping the vehicle 1 meter before
the boundary in one spatial dimension. This logic must satisfy three
criteria: (1) allow unrestricted movement when the vehicle is far from the
boundary, (2) take into account limitations on maximum possible braking
force of the system, and (3) bring the vehicle smoothly to a stop at the 1
meter buffer.

The module does so by restricting the velocity as a function of the
vehicle's distance from the boundary. Given the maximum braking force of
the systems, a natural approach is to limit the velocity as a function of
the \emph{square root} of the distance from the boundary. However, such a
limitation results in a discontinuity in control at the boundary, as
experienced by the controllers in Chapters~\ref{chap:memo15}
and~\ref{chap:emsoft16}. In control theory terminology, such a system
behaves as a bang-bang controller, which allows maximum positive
acceleration until the last possible moment, when maximum braking force is
applied. Such controllers suffer from oscillations at the boundary,
violating the third criteria (smoothness).

A solution developed by the Ardupilot engineers is to apply a piecewise
limitation to velocity -- far from the boundary enforce a square root
relationship between position and velocity, while close to the boundary,
enforce a linear relationship. Such a linear relationship removes the
control discontinuity at the boundary. Section~\ref{sec:proof-geofence}
provides the formal details of the velocity restriction.

The actual system passes the restricted velocity to an underlying velocity
controller. In this chapter, we ignore the dynamics of the inner loop
velocity controller controller and as in the rest of this dissertation,
treat the system as a double integrator, in which the control law outputs
acceleration. Such an approximation is an important first step towards
developing new formal verification techniques, as it still forces us to
handle the non-trivial velocity restriction law.  We leave composition with
the velocity controller as an interesting direction for future work.

Given the double integrator model and the velocity restriction law, we show
how to derive the controller's logic to guarantee the velocity restriction
and thus enforce the boundary on position. This logic comes in the form of
state-dependent upper bounds on the control signal (acceleration). The
actual control law implementation takes the minimum of these constraints
and the pilot's desired acceleration. Crucially, we derive these
constraints using our barrier certificate theorem for sampled-data systems,
which allows us to derive the constraints assuming a continuous time
controller, and transfer these results to the sampled-data model by proving
two simple side-conditions on the intersample system behavior.

Section~\ref{sec:model} presents the formal logic we use to reason about
sampled-data systems and gives a double integrator model of the geofence
within this logic. Section~\ref{sec:proof-barrier} provides formal details
on barrier certificates for sampled-data systems and
Section~\ref{sec:proof-geofence} shows how to apply them to reason about
the geofence. Section~\ref{sec:coq} describes the Coq implementation of our
results with Section~\ref{sec:differentiation} describing how we extend
formal verification using barrier certificate to piecewise functions.

\section{Logic}
\label{sec:model}
This chapter uses a different formal framework than the previous
chapters. Rather than linear temporal logic, we use a logic over system
trajectories, i.e. functions $\trajectory{n}$ from time to state.

Our logic is defined in terms of two relations: models (written $F \models
P$) expresses that a trajectory predicate $P$ holds on a trajectory $F$;
and entails (written $P \entails Q$) expresses that a trajectory predicate
$Q$ holds on all trajectories that $P$ holds on.

Formally, entailment is defined as follows:
\begin{definition}[Entailment]
For trajectory predicates $P$ and $Q$,
\[
P \entails Q \defeq \forall F \in \trajectory{n} : F \models P \implies F \models Q
\]
\label{def:entailment}
\end{definition}

We now define the models relation for several types of predicates. First,
we define models for a predicate that expresses whether a trajectory is a
valid solution to a sampled-data system whose intersample dynamics are
given by $\dt{\vecbold{z}} = f(\vecbold{z},u(\vecbold{z_k}))$ where
$\vecbold{z_k}$ is the state at the last sample and $u \in \R^n \rightarrow
\R^m$ is the control law. This definition requires a notion of well-formed
sample times:
\begin{definition}[Sample times]
A sampling sequence $t_k$ with $k \in \N$ is well-formed with bound $\Delta$ if
\[\begin{array}{lr}
t_0 = 0 & \wedge \\
\forall k \in \N : 0 < t_{k + 1} - t_k \leq \Delta & \wedge \\
\forall t \in \Rpos : \exists k \in \N : t_k \leq t < t_{k + 1}
\end{array}
\]
\end{definition}
Using the definition of well-formed sample times, we can formally specify
the models relation for a sampled-data system whose sample times (time
between discrete controller executions is bounded by $\Delta$:
\begin{definition}[Sampled data solutions]
\begin{align*}
F \models \evolvessmpl{\dt{\vecbold{z}} = f(\vecbold{z},u)}{\Delta}
\defeq&~\exists t_k \in \{ t_k : t_k \text{ well-formed} \} : \\
&\qquad \forall k \in \N : \forall t \in [t_k,t_{k+1}) : \dt{F}(t) = f(F(t),u(F(t_k)))
\end{align*}
\label{def:solution-smpl}
\end{definition}
In Definition~\ref{def:solution-smpl}, we enforce more structure on the
system than our \SysA{} abstraction from previous chapters
(Definition~\ref{def:sys-abstraction}). In particular, we make explicit the
control law and enforce that the control input is held constant between
samples. Adding this structure allows us to state a powerful barrier
function theorem for sampled-data systems.

Definition~\ref{def:solution-smpl} does not put any restrictions on the
initial state of the trajectory, $F(0)$. For this, we use state
predicates -- predicates over a single state. A state predicate holds on
a trajectory if it holds on the initial state of that trajectory:
\begin{definition}[Initially]
For a state predicate $P$ and a trajectory $F$,
\[
F \models P \defeq F(0) \models P
\]
\label{def:init}
\end{definition}

Next, we need to express that a state predicate is an invariant of a
trajectory. For this, we define an always operator, which takes a state
predicate and expresses that it holds for all time along a trajectory.
\begin{definition}[Always]
For a state predicate $P$ and a trajectory $F$,
\[
F \models \Always{P} \defeq \forall t \in \Rnneg : F(t) \models P
\]
\label{def:always}
\end{definition}

Finally, sampled-data systems often have an upper bound on the time between
samples. Thus, we define a bounded always operator in order to reason about
invariants between samples. This operator takes a state predicate and
expresses that it holds until a time bound $\Delta$ along a trajectory.
\begin{definition}[Bounded Always]
For a state predicate $P$, a constant $\Delta \in \Rnneg$, and a trajectory $F$,
\[
F \models \AlwaysT{\Delta}{P} \defeq \forall 0 \leq t \leq \Delta : F(t) \models P
\]
\label{def:bounded-always}
\end{definition}

\section{Exponential barrier certificates}
\label{sec:proof-barrier}
Deductive approaches to hybrid system verification typically involve a
continuous analogue of induction. Barrier certificates are one such
technique. In the typical formulation of barrier certificates, one proves
the invariance of $B(\vecbold{z}) \leq 0$ by proving that the value of
$B(\vecbold{z})$ does not increase along trajectories of the system. Recent
work from control theory~\cite{kong2013barrier} has relaxed this
requirement -- the rate of change of $B(\vecbold{z})$ along trajectories
must be proportional to its value. This allows the value of
$B(\vecbold{z})$ to increase along trajectories, but requires that the rate
of change slow as trajectories approach the boundary $B(\vecbold{z}) =
0$. Essentially, $B(\vecbold{z})$ can exponentially decay towards the
boundary but never cross it.

In this section, we present two barrier certificate proof rules for
sampled-data systems.  The first (Lemma~\ref{thm:barrier-smpl-weak}) is a
trivial adaptation of the rule from~\cite{kong2013barrier} to sampled-data
systems. We explain why this rule, by itself, is insufficient.  Then we
present our new version of exponential barrier certificates for
sampled-data systems that decomposes verification into an exponential decay
property of the continuous time system approximation and two simple
properties about the intersample behavior
(Theorem~\ref{thm:barrier-smpl}). In Section~\ref{sec:proof-geofence}, we
show how to use these proof rules to reason about the Ardupilot
controller. All results in this section have been formalized in the Coq
proof assistant. To the best of our knowledge, this is the first such
formalization of exponential barrier certificates.

Barrier certificates are functions $B \in \R^n \rightarrow \R$ that assign
a scalar to every state. One establishes the invariance of $B(\vecbold{z})
\leq 0$ by proving a property about the rate of change of $B(\vecbold{z})$
along trajectories of the system. Formally, $\nabla B \dotprod
f(\vecbold{z},\vecbold{z_k})$ gives the rate of change or time derivative
of $B$ along trajectories of the system, also known as the Lie derivative
of $B$. Here, $\nabla B$ denotes the gradient of $B$, or vector of partial
derivatives $[\frac{\partial B}{\partial x_1},\ldots,\frac{\partial
    B}{\partial x_n}]$.

Lemma~\ref{thm:barrier-smpl-weak} gives our trivial adaptation of barrier
certificates from~\cite{kong2013barrier} to sampled-data systems.

\begin{lemma}
Consider a continuously differentiable function $B \in \R^n \rightarrow
\R$, constants $\lambda \in \R$ and $\Delta \in \Rpos$, and a state
predicate $P$. If the following condition holds:
\[
\forall \vecbold{z}, \vecbold{z_k} \in \{\vecbold{y} \in \R^n~|~\vecbold{y} \models P\} : \nabla B \dotprod f(\vecbold{z},\vecbold{z_k}) \leq \lambda B(\vecbold{z})
\]
then
\[
\begin{array}{cl}
&
B(\vecbold{z}) \leq 0~\wedge~
\Always{P}~\wedge~
\evolvessmpl{\dt{\vecbold{z}} = f(\vecbold{z},u(\vecbold{z_k}))}{\Delta} \\
\entails
&
\Always{B(\vecbold{z}) \leq 0}
\end{array}
\]
\label{thm:barrier-smpl-weak}
\end{lemma}
% TODO: Can we claim novelty here by putting P into the theorem. This is
% essentially a combination of differential cut and the exponential
% condition, but doesn't seem to have been stated together anywhere.

Lemma~\ref{thm:barrier-smpl-weak} states that if $B(\vecbold{z}) \leq 0$
holds initially, and if, assuming a known invariant $P$, the time
derivative of $B$ is at most proportional to its value, then
$B(\vecbold{z}) \leq 0$ is an invariant.

While this theorem is powerful for continuous time systems, it is lacking
for sampled-data systems. The problem is that the premise of the theorem
does not constrain the relationship between the sampled state
$\vecbold{z_k}$ and the current state $\vecbold{z}$. However, as we will
see, this theorem still has utility. In particular, once an invariant $P$
has been established using Theorem~\ref{thm:barrier-smpl},
Lemma~\ref{thm:barrier-smpl-weak} can be used to establish a new invariant
using the invariance of $P$.

We would like the premise of Theorem~\ref{thm:barrier-smpl-weak} to be
relaxed so that the control input $u$ is applied continuously. In other
words, we would like to only prove $\nabla B \dotprod
f(\vecbold{z},\vecbold{z}) \leq \lambda B(\vecbold{z})$ rather than $\nabla
B \dotprod f(\vecbold{z},\vecbold{z_k}) \leq \lambda B(\vecbold{z})$. We
can achieve this relaxation if we can establish an additional condition on
the intersample behavior: the time derivative of $B$ does not change by
more than a constant $C$ between samples. Under such a condition, the
property $B(\vecbold{z}) \leq 0$ is not an invariant, but $B(\vecbold{z})
\leq C\cdot \Delta$ is. This mirrors a control design practice -- treat the
controller as continuous and add a constant safety buffer to account for
the approximation error.

This argument is formalized in
Theorem~\ref{thm:barrier-smpl}. Condition~\eqref{thm:barrier-smpl-exp} is
the relaxed version of the premise of
Theorem~\ref{thm:barrier-smpl-weak}. Condition~\eqref{thm:barrier-smpl-inter}
characterizes the behavior of the state between samples, while
condition~\eqref{thm:barrier-smpl-deriv-change} bounds the change in time
derivative of $B$, under the characterization established
by~\eqref{thm:barrier-smpl-inter}. Such a decomposition allows us to use
any continuous reasoning technique to
dispatch~\eqref{thm:barrier-smpl-inter} and arithmetic reasoning to
establish~\eqref{thm:barrier-smpl-deriv-change}.

\begin{theorem}
Consider a continuously differentiable function $B \in \R^n \rightarrow
\R$, constants $C,\Delta \in \Rpos$, a state predicate $P$, and a state
relation $S$. If the following conditions hold:
\begin{enumerate}[label=\roman*), ref=\roman*]
\item
\label{thm:barrier-smpl-exp}
$\forall \vecbold{z} \in \{\vecbold{y} \in \R^n~|~\vecbold{y} \models P\} : \nabla B \dotprod f(\vecbold{z},\vecbold{z}) \leq -\frac{B(\vecbold{z})}{\Delta}$
\item
\label{thm:barrier-smpl-inter}
$\forall \vecbold{z_k} \in \R^n : \vecbold{z} = \vecbold{z_k} \wedge \evolvessmpl{\dt{\vecbold{z}} = f(\vecbold{z},\vecbold{z_k})}{\Delta} \entails \AlwaysT{\Delta}{(\vecbold{z}_k,\vecbold{z}) \in S}$
\item
\label{thm:barrier-smpl-deriv-change}
$\forall (\vecbold{z_k}, \vecbold{z}) \in S : \nabla B \dotprod f(\vecbold{z},\vecbold{z_k}) \leq \max{(\nabla B \dotprod f(\vecbold{z_k},\vecbold{z_k}) + C,0)}$
\end{enumerate}
then
\[
\begin{array}{cl}
&
B(\vecbold{z}) \leq C \cdot \Delta~\wedge~
\Always{P}~\wedge~
\evolvessmpl{\dt{\vecbold{z}} = f(\vecbold{z},u(\vecbold{z_k}))}{\Delta} \\
\entails
&
\Always{B(\vecbold{z}) \leq C \cdot \Delta}
\end{array}
\]
\label{thm:barrier-smpl}
\end{theorem}
%TODO dB must be continuous
\begin{proof}
For any $k$ and for any $t \in [t_k, t_{k+1})$,
\begin{align}
\nonumber
B(\vecbold{z}(t)) - B(\vecbold{z}(t_k)) &= \int_{t_k}^t \nabla B \dotprod f(\vecbold{z}(\tau),\vecbold{z}(t_k)) d\tau\\
\tag{by~\eqref{thm:barrier-smpl-deriv-change} and~\eqref{thm:barrier-smpl-inter}}
&\leq \int_{t_k}^t \big(\max{(\nabla B \dotprod f(\vecbold{z}(t_k),\vecbold{z}(t_k)) + C,0)}\big) d\tau \\
\tag{by~\eqref{thm:barrier-smpl-exp}}
&\leq \int_{t_k}^t \left(\max({-\frac{B(\vecbold{z}(t_k))}{\Delta} + C,0)}\right) d\tau \\
\nonumber
&= (t - t_k) \cdot \left(\max{(-\frac{B(\vecbold{z}(t_k))}{\Delta} + C,0)}\right)
\end{align}

Therefore, for any $k$ and any $t \in [t_k,t_{k+1})$, since $t - t_k \leq Delta$,
\begin{equation}
B(\vecbold{z}(t_k)) \leq C\cdot \Delta \implies B(\vecbold{z}(t)) \leq C\cdot \Delta
\end{equation}
By induction on $k$, $\forall t \geq 0 : B(\vecbold{z}(t)) \leq C\cdot \Delta$
\end{proof}

Note the $\max{}$ in condition~\eqref{thm:barrier-smpl-deriv-change} of
Theorem~\ref{thm:barrier-smpl}. The intersample derivative of $B$ does not
need to be close to the sample time derivative of $B$ as long as it is
non-positive. This subtlety bares a resemblance to the normal formulation
of barrier certificates in which the time derivative of $B$ must always be
non-positive.

\section{Ardupilot controller}
\label{sec:proof-geofence}
Now that we have established the general theory, we can use it to reason
about a double integrator model of the Ardupilot controller, whose
intersample behavior is given by:
\[
\dt{x} = v, \dt{v} = u(x_k,v_k)
\]
The control law $u$ is designed to enforce $x \leq 0$.  We begin this
section by describing how $u$ is designed to (almost) achieve this
objective. As mentioned in Section~\ref{sec:exp-smpl-overview}, the control
law design comes in the form of state-dependent upper bounds on the control
signal $u$. As described in that section, the actual control law
implementation takes the minimum of the pilot's desired acceleration and
those constraints.

The Ardupilot control law was designed assuming that the law is applied
continuously (not as a sampled-data system). This approximation simplifies
design and reasoning but also introduces an error, which actually results
in a violation of $x \leq 0$. We will ultimately apply
Theorem~\ref{thm:barrier-smpl} to quantify this violation.

Relevant to the controller design are the physical constraints on the
system, particularly the maximum possible acceleration. Since we are
interested in a controller that stops the system's position from violating
some boundary, the important physical constraint is a limit on the system's
braking acceleration. That is, we need to ensure that the control law does
not require deceleration of a greater magnitude than is physically
possible. We denote by $\umax$ this magnitude -- we need to ensure that the
system never reaches a state $x,v$ in which $u(x,v) < -\umax$.

Since the stopping distance for a particle with acceleration $a$ and
initial velocity $v$ is $\frac{v^2}{2a}$, the $\umax$ constraint implies
that the system must enforce the following invariant:
\[
x \leq \frac{\max{(v,0)}^2}{2\umax}
\]
However, since the controller was designed under a continuous time
approximation, the invariant needs to be strengthened by using a more
conservative braking acceleration $\umaxc < \umax$. At the end of this
section, we will give precise conditions on $\umaxc$ to ensure that the
system never reaches a state $x,v$ in which $u(x,v) < -\umax$. The
strengthened invariant is thus:
\[
x \leq \frac{\max{(v,0)}^2}{2\umaxc}
\]
This invariant is still insufficient. The problem is that such an invariant
will result in a control discontinuity at the position boundary $x =
0$. This occurs if the system is in a state in which $x =
\frac{\max{(v,0)}^2}{2\umaxc}$. Such a state requires that the controller
issue constant acceleration $-\umaxc < 0$ when $x < 0$ and permits zero
acceleration when $x = 0$. As described in the introduction, such a control
law produces undesirable oscillations at the boundary and is physically
impossible.

To solve this problem, the Ardupilot developers designed a more
conservative invariant that is a piecewise function of the distance from
the boundary. Close to the boundary, the velocity limit is linear in the
distance, while farther away, the velocity limit is proportional to the
square root of the distance. This relationship is depicted in
Figure~\ref{fig:sqrt-lin}. Notice that the piecewise transition between
these two relationships occurs where they are tangent. This ensures that
(a) a barrier certificate describing the region is continuously
differentiable, and (b) the control constraints induced by such a barrier
are continuous.

\begin{figure}
\centering
\begin{tikzpicture}
      \fill [lightgray, domain=0:2.5, variable=\x]
        (-0.5,-2.5)
        -- plot ({-\x*\x/2 - 1/2}, {\x})
        -- (-3.625, -2.5)
        -- cycle;
      \fill [lightgray, domain=-1:2.5, variable=\x]
        (-1, -2.5)
        -- plot ({\x}, {-\x})
        -- cycle;
      \draw[->] (-3,0) -- (3,0) node[right] {$x$};
      \draw[->] (0,-3) -- (0,3) node[above] {$v$};
      \draw[domain=0:2.5,smooth,variable=\v,red]  plot ({-\v*\v/2 - 1/2},{\v})
          node[left] at (-3.625,2.5) {$x + \frac{\umaxc \p^2}{2} + \frac{v^2}{2\umaxc} \leq 0$};
      \draw[domain=-2.5:2.5,smooth,variable=\x,blue]  plot ({\x},{-\x}) node[right] at (2.5,-2.5) {$\p v + x \leq 0$};
      \draw[black,variable=\x,domain=-3:2.5,dashed] plot ({\x},{1}) node[right] at (2.5,1) {$v = \umaxc \p$};
\end{tikzpicture}
\caption{A depiction of $\Bpw (x,v) \leq 0$ where the red curve represents the first branch of the piecewise function and blue the second.}
\label{fig:sqrt-lin}
\end{figure}

Formally, the region depicted in Figure~\ref{fig:sqrt-lin} is $\Bpw (x,v)
\leq 0$, where
\begin{equation}
\Bpw (x,v) =
\begin{cases}
\p v + x & \text{if } v \leq \umaxc\p \\
x + \frac{\umaxc \p^2}{2} + \frac{v^2}{2\umaxc} & \text{otherwise}
\end{cases}
\label{eq:barrier-piecewise-dbl-int}
\end{equation}
Our goal is to (almost) enforce the invariance of $x \leq 0$ -- that is, we
would like to enforce $x \leq K$ for some positive constant $K$.  We now
take a two step process to prove the invariance of $x \leq K$ for some
constant $K$. First, we use Theorem~\ref{thm:barrier-smpl} and constraints
on $u$ to prove the invariance of $\Bpw (x,v) \leq K$. Then we use
Lemma~\ref{thm:barrier-smpl-weak} and the invariance of $\Bpw (x,v) \leq K$
to prove the invariance of $x \leq K$. The assumptions on $u$ are as
follows:
\begin{assumption}
For any $x,v \in \R$,
\[
u(x,v) \leq
\begin{cases}
\frac{-(\p v + x)}{T\p} - \frac{v}{\p} & \text{if } v \leq \umaxc\p \\
\frac{-\umaxc x}{T v} - \frac{\umaxc^2 \p^2}{2Tv} - \frac{v}{2T} - \umaxc & \text{otherwise}
\end{cases}
\]
\label{ass:u_pw_bound}
\end{assumption}

Note that the division by $v$ in the second branch of
assumption~\ref{ass:u_pw_bound} is not problematic, since in that branch,
$v > \umaxc\p > 0$.

Since the control law was designed using a continuous time approximation,
we need to put a constant upper bound on the control in order to bound the
intersample behavior and thus apply Theorem~\ref{thm:barrier-smpl}. Such a
bound is captured in the following assumption:
\begin{assumption}
$\forall x, v \in \R : u(x,v) \leq \umaxc$
\label{ass:u_umax_bound}
\end{assumption}

Theorem~\ref{thm:barrier-smpl} requires reasoning about the time derivative
of $\Bpw$, which is given by~\eqref{eq:dbarrier-piecewise-dbl-int}. The
formal details details of reasoning about Lie derivatives of piecewise
functions, like the one in~\eqref{eq:dbarrier-piecewise-dbl-int} are
deferred to Section~\ref{sec:differentiation}.
%TODO explain why we quantify over v and v_k but not x and x_k
\begin{equation}
\forall v, x_k, v_k \in \R : \nabla \Bpw \dotprod [v,u(x_k,v_k)] =
\begin{cases}
\p u(x_k,v_k) + v & \text{if } v \leq \umaxc\p \\
v + \frac{v\cdot u(x_k,v_k)}{\umaxc} & \text{otherwise}
\end{cases}
\label{eq:dbarrier-piecewise-dbl-int}
\end{equation}

The next three lemmas satisfy the three conditions of
Theorem~\ref{thm:barrier-smpl} for $\Bpw$.

\begin{lemma}[Condition~\eqref{thm:barrier-smpl-exp}]
For any $x,v \in \R$,
\[
\nabla \Bpw \dotprod [v,u(x,v)] \leq -\frac{\Bpw (x,v)}{\Delta}
\]
\label{lem:exp-condition-dbl-int}
\end{lemma}
\begin{proof}
Solving the above in equality for $u$ results in exactly the inequality in
assumption~\ref{ass:u_pw_bound}.
\end{proof}

\begin{lemma}[Condition~\eqref{thm:barrier-smpl-inter}]
For any $x_k, v_k \in \R$,
\[
\begin{array}{cl}
& x = x_k~\wedge~v=v_k~\wedge~\evolvessmpl{\dt{x}=v,\dt{v}=u(x_k,v_k)}{\Delta} \\
\entails
& \AlwaysT{\Delta}{v \leq v_k + \max{(u(x_k,v_k), 0)}\cdot \Delta}
\end{array}
\]
\label{lem:intersmpl-dbl-int}
\end{lemma}
\begin{proof}
By integration of $v$ from 0 to $\Delta$.
\end{proof}

The intersample dynamics are sufficiently simple that
Lemma~\ref{lem:intersmpl-dbl-int} can be proven by integration. However,
since condition~\eqref{thm:barrier-smpl-inter} of
Theorem~\ref{thm:barrier-smpl} concerns purely continuous time dynamics,
the intersample behavior of more complex systems can be verified using a
number of powerful techniques from prior work, including differential
induction~\cite{Platzer10DAL}. Making such a connection formal is an
interesting direction for future work.

\begin{lemma}[Condition~\eqref{thm:barrier-smpl-deriv-change}]
For any $x_k, x, v_k, v \in \R$ such that $v \leq v_k + \max{(u(x_k,v_k),
  0)}\cdot \Delta$,
\[
\nabla \Bpw \dotprod [v,u(x_k,v_k)] \leq \nabla \Bpw \dotprod [v_k,u(x_k,v_k)] + \violation
\]
\label{lem:deriv-change-dbl-int}
\end{lemma}
\begin{proof}
By assumption~\ref{ass:u_umax_bound} and first order reasoning over real
arithmetic.
\end{proof}

It is important to note that the above theorem only relies on the very
simple intersample behavior of $v$ and constant bounds on $u$. It does not
rely on bounds on position nor does it depend on the more complex bounds on
$u$ used in Lemma~\ref{lem:exp-condition-dbl-int}. This serves as evidence
that Theorem~\ref{thm:barrier-smpl} allows one to prove relatively simple
side conditions in order to transfer continuous time results to the
sampled-data domain.

Given Lemmas~\ref{lem:exp-condition-dbl-int},~\ref{lem:intersmpl-dbl-int},
and~\ref{lem:deriv-change-dbl-int}, we can now prove the invariance of
$\Bpw (x,v) \leq (\violation) \cdot \Delta$.

\begin{lemma}
\[
\begin{array}{clc}
&
\Bpw (x,v) \leq (\violation) \cdot \Delta & \wedge \\
& \evolvessmpl{\dt{x}(t) = v(t), \dt{v}(t) = u(x(t_k),v(t_k))}{\Delta} & \\
\entails
&
\Always{\Bpw (x,v) \leq (\violation) \cdot \Delta} &
\end{array}
\]
\label{lem:barrier-piecewise-dbl-int}
\end{lemma}
\begin{proof}
By
Theorem~\ref{thm:barrier-smpl}. Lemmas~\ref{lem:exp-condition-dbl-int},~\ref{lem:intersmpl-dbl-int},
and~\ref{lem:deriv-change-dbl-int} satisfy
conditions~\eqref{thm:barrier-smpl-exp},~\eqref{thm:barrier-smpl-inter},
and~\eqref{thm:barrier-smpl-deriv-change} of
Theorem~\ref{thm:barrier-smpl}, respectively. We instantiate the state
predicate $P$ of Theorem~\ref{thm:barrier-smpl} with the trivial predicate
\True.
\end{proof}

Finally, given the invariance of $\Bpw (x,v) \leq (\violation) \cdot \Delta$
provided by Lemma~\ref{lem:barrier-piecewise-dbl-int}, we can return to
Theorem~\ref{thm:x-bound-dbl-int}, which establishes the invariance of $x
\leq (\violation) \cdot \Delta$. We restate the theorem here for convenience:

\begin{theorem}
\[
\begin{array}{clc}
&
\Bpw (x,v) \leq (\violation) \cdot \Delta & \wedge \\
&
x \leq (\violation) \cdot \Delta & \wedge \\
&
\evolvessmpl{\dt{x} = v, \dt{v} = u(x_k,v_k)}{\Delta} & \\
\entails
&
\Always{x \leq (\violation) \cdot \Delta} &
\end{array}
\]
\label{thm:x-bound-dbl-int}
\end{theorem}
\begin{proof}
We apply Theorem~\ref{thm:barrier-smpl-weak} with barrier function
\[
\Bx (x,v) = x - (\violation)\cdot \Delta\]
constant $\frac{-1}{\p}$, and state predicate
\[
\Bpw (x,v) \leq (\violation) \cdot \Delta
\]
Simple arithmetic reasoning reveals
that for all $x,v \in \R$,
\begin{align}
\nabla \Bx \dotprod [v,\ignorearg] + \frac{\Bx (x,v)}{\p} &\leq \Bpw (x,v) - (\violation)\cdot \Delta \\
&\leq 0
\end{align}
which satisfies the premise of Theorem~\ref{thm:barrier-smpl-weak}.
\end{proof}

Theorem~\ref{thm:x-bound-dbl-int} verifies the design procedure followed by
the Ardupilot engineers -- the controller can be designed assuming that it
is applied continuously and a small constant buffer can compensate for the
approximation error. Moreover, it shows that the constraints on $u$ fall
out naturally from the velocity bounds depicted in
Figure~\ref{fig:sqrt-lin} by a simple application of
Theorem~\ref{thm:barrier-smpl}.

Finally, we provide conditions on $\umaxc$ to ensure that the system never
reaches a state $x,v$ in which $u(x,v) < -\umax$. In other words, for any
$x$ and $v$ satisfying $\Bpw (x,v) \leq (\violation) \cdot \Delta$ and $x \leq
(\violation) \cdot \Delta$, we need to ensure that there exists a value
satisfying the upper bound from assumption~\ref{ass:u_pw_bound} and the
lower bound $-\umax \leq u(x,v)$. Arithmetic reasoning reveals that the
constraints are:
\begin{equation}
\umaxc \leq \frac{\umax}{1 + \frac{2\Delta}{\p}}
\end{equation}
For Ardupilot, the sample time is small relative to $\p$, so $\umaxc$ is
close to $\umax$. This again verifies the design procedure followed by the
Ardupilot engineers -- to account for error, the controller is designed
with a conservative braking acceleration $\umaxc$ rather than the actual
physical limit $\umax$.

\section{Coq implementation}
\label{sec:coq}
We have implemented all of the results of this chapter in the Coq proof
assistant. The development is available from:
\url{https://github.com/dricketts/barrier-sampled-data}. The core theory
required 598 lines of definitions and 620 lines of proof, while the
geofence model required 99 lines of definitions and 205 lines of proof. On
the other hand, the Second-Derivative Controller from
Chapter~\ref{chap:emsoft16}, verified without the barrier function theorems
of this chapter, required 484 lines of proof, despite simpler constraints
on the control law. This provides evidence that
Theorem~\ref{thm:barrier-smpl} and Lemma~\ref{thm:barrier-smpl-weak} have
the potential to considerably reduce the formal proof burden for safety of
sampled-data systems.
%TODO update these numbers if proof dev changes

\subsection{Differentiation}
\label{sec:differentiation}
In Section~\ref{sec:proof-geofence}, we glossed over the formal details of
verifying the gradient of a barrier function. Working in an expressive
proof assistant like Coq allows us to verify this step as well, even for
piecewise functions. That is, we can formally prove the validity of
equations like~\eqref{eq:dbarrier-piecewise-dbl-int} using
Lemma~\ref{lem:pw-deriv}:

\begin{lemma}
For functions $e_1, e_2, e_3, e_4 \in \R^n \rightarrow \R$, variable $x$,
vector $f \in \R^n$, and constant $c \in \R$, if the following conditions
hold,
\begin{enumerate}[label=\roman*), ref=\roman*]
\item $x = c \implies e_1 = e_2$
\item $x = c \implies e_3 = e_4$
\item $x \leq c \implies \nabla e_1 \dotprod f = e_3$
\item $x \geq c \implies \nabla e_2 \dotprod f = e_4$
\end{enumerate}
then
\[
\nabla \left(
\begin{cases}
e_1 & \text{if } x \leq c \\
e_2 & \text{otherwise}
\end{cases}
\right)
\dotprod
f
=
\begin{cases}
e_3 & \text{if } x \leq c \\
e_4 & \text{otherwise}
\end{cases}
\]
\label{lem:pw-deriv}
\end{lemma}

To the best of our knowledge, this is the first formal verification
framework for hybrid systems that can handle piecewise functions.

\subsection{Logics in Coq}
%TODO should we talk about benefit of shallow encoding?
We implemented our trajectory logic using Charge!~\cite{chargecoregit}, a
framework for easily defining and reasoning about logics within
Coq. Charge! allows a us to declare that a particular type forms a logic by
proving that a small set of axioms hold for that type. In our case, these
axioms are proved automatically. In return, we automatically get standard
logical operators such as conjunction, disjunction, and implication lifted
into the logic. For example, in Section~\ref{sec:proof-barrier}, we
frequently wrote formulas like $P \wedge Q$ where $P$ and $Q$ are
trajectory predicates. Without Charge!, we would have to define what it
means to conjoin two trajectory predicates. With Charge!, we get this for
free.

Along with the normal logic definitions, Charge! also automatically
provides standard logical proof rules like commutativity of conjunction and
tactics to discharge simple proof obligations. In short, Charge! provides
much of the boiler-plate logical reasoning and definitions for free.

Most significantly, Charge! allows us to easily use the most appropriate
logic for each proof obligation, all within the same formal foundation of
Coq. For example, condition~\eqref{thm:barrier-smpl-inter} of
Theorem~\ref{thm:barrier-smpl} is currently phrased as a trajectory
predicate entailment. However, Platzer's differential dynamic logic
(\dL{})~\cite{Platzer15substitution} may be a better logic for reasoning
about this condition. Charge! could allow us to easily build \dL{} as a
logic in Coq and re-phrase this condition in terms of \dL{}.

%% \section{Conclusion}
%% \label{sec:conclusion}
%% The difficulty of hybrid system verification demands that we have a variety
%% of higher-order proof rules at our disposal. This work adds one such
%% powerful tool to the arsenal for the important class of sampled-data
%% systems. Moreover, we show how leveraging the expressiveness of a
%% higher-order proof assistant (Coq) allows us to extend the set of operators
%% that are in scope for formal verification using barrier certificates.

% Talk about how existing work is great for general hybrid systems, but we
% focus on writing higher order theorems that leverage the structure of
% particular classes of hybrid systems, in our case sampled-data systems.

% Do we want to talk about number of lines of proof, development effort, etc.
% Should we compare to development effort for 1D pos from emsoft paper?
% It took just a few days to verify.

\section{Acknowledgments}
A special thanks to the Ardupilot developers, particularly Leonard Hall,
who designed the piecewise velocity restriction that we modeled in this
chapter.

This chapter, \expsmplack{}


\chapter{Related Work}
\label{chap:related}
There has been a tremendous amount of work on the specification and
verification of hybrid systems, both from the verification community and
the control theory community.  In this section, we describe some of the
prior work in this area, highlighting the commonalities and difference with
our own work.

\section{Hybrid Automata}

\section{Deductive logics}

\subsection{Temporal Logic}
The foundation for Chapters~\ref{chap:memo15} and~\ref{chap:emsoft16} is
temporal logic, and there has been a lot of work on composition in temporal
logic, most notably by Abadi and
Lamport~\cite{abadi1995conjoin,abadi1994realtime}.  Their work describes
how to reason about the conjunction of LTL specifications, but they do not
deal with the interplay between conjunction and \progress{} or substitution
and \progress{}.  It is important to note that the preservation/\progress{}
split is not the same as the safety/liveness split as both preservation and
progress are safety properties.  Abadi and Lamport address Zeno
specifications in~\cite{abadi1994realtime}, but do not address the
relationship between Zenoness and conjunction.  Finally, neither of these
works addresses substitution in the presence of progress.  It is also
important to note that we use disjunction for non-deterministic choice
between controllers while much of the work in temporal logic uses
disjunction to represent interleaving specifications of asynchronous
concurrent systems.  Other work includes techniques for decomposing LTL
verification into a search for suitable barrier
certificates~\cite{wongpiromsarntemporal15} and deductive rules for
synthesizing controllers satisfying ATL* properties~\cite{dimitrovaATL14}.
While these approaches apply to more temporal logic formulas than ours,
they do not address verification using conjunction, disjunction, and
substitution.  There has also been work on synthesis using approximate
bisimulations~\cite{tabuada08approx}.  We focus on the complementary task
of composing and reusing controllers.

\section{Architectures for Cyber-physical Systems}
In Chapters~\ref{chap:memo15} and~\ref{chap:emsoft16}, we describe how to
architect hybrid systems in order to ensure safety in the presence of
complex, unverified controllers.  There has been prior work in this area,
much of it based on the simplex architecture~\cite{sha1996evolving}.  In
this architecture, there is a simple module that constantly monitors the
system and takes control away from more complex modules before the system
can enter an unsafe state.  We follow a similar principle with our
controller architecture from those two chapters. Our work complements that
based on~\cite{sha1996evolving} by formally verifying the simple monitoring
module (our controllers) rather than relying on their simplicity for
correctness.

Livadas and Lynch solve a similar problem using hybrid I/O automata to
model and reason about ``protectors'' for hybrid systems~\cite{LivadasL98}.
A protector is designed to ensure a safety property of a particular hybrid
system. Livadas and Lynch present a series of rules for conjunctive
composition of protectors.  In contrast, our work also supports other forms
of composition, \textit{e.g.} disjunctive composition, and we show how
these operators can be \emph{combined} to achieve modular verification.

Our theory from Chapter~\ref{chap:emsoft16} is closely related to the work
on modular construction of nonblocking supervisory controllers in
discrete-event systems~\cite{wonham1988supervisory-small}.  However, our
models explicitly include differential equations rather than requiring a
discrete abstraction and do not require a notion of termination in a marked
state.  Moreover, to the best of our knowledge, none of this work makes the
inductive invariant explicit and separates preservation from \progress{}.

Finally, the area of switching control~\cite{liberzon2012switching} from
control theory focuses on (in our terminology) disjunctive composition of
controllers.  Our inclusion of invariants in the discrete controller
corresponds to expressing the switching boundary in a switching controller.
However, the focus of switching control theory is optimality, stability,
and transitionability, whereas we focus on complementary properties like
bounding the state space.
%TODO read more about switching control

\section{Inductive Methods for Hybrid Systems}

\section{Sampled-data Systems}


\chapter{Conclusions and Future Work}
\label{chap:conclusion}
We have presented a series of proof rules for safety verification of
sampled-data systems and evaluated them on various versions of the
sampled-data double integrator, an important benchmark in control
theory. Chapter~\ref{chap:memo15} gave rules for discrete induction
specialized to sampled-data systems and acyclic composition of
components. Chapter~\ref{chap:emsoft16} described a general framework for
building sampled-data systems using conjunction (parallel composition),
disjunction (alternative composition), and substitution (reuse). Finally
Chapter~\ref{chap:exp-smpl} revisited an atomic building block from
previous chapters, using a better performing but more complex position
bounding controller to motivate a stronger continuous induction theorem for
sampled-data systems.

All of the work described in this dissertation was done in the expressive
and foundational context of the Coq proof assistant. This expressiveness
was essential, as it allowed us to add domain-specific reasoning techniques
as theorems and dispatch side conditions using formal proofs, rather than
using axioms or tactics and dispatching side conditions using unverified
decision procedures. In summary, the work described in this document was in
support of the following:

\thesis

There are a number of exciting future research directions for this work. In
the rest of this section, we discuss some of these directions.

\paragraph{Horizontal composition}
Chapter~\ref{chap:emsoft16} explored composition of sampled-data systems,
particularly those controlling overlapping sets of actuators. However, the
proof rules of Chapter~\ref{chap:emsoft16} were only evaluated on the
simpler but poorer performing atomic position (height) controller from
Chapter~\ref{chap:memo15}, not on the more complex but better performing
position controller from Chapter~\ref{chap:exp-smpl}. It would be valuable
to explore composition and re-use of the more complex controller from
Chapter~\ref{chap:exp-smpl}, as this would be an important step towards
verification of the full Ardupilot geofencing module. An promising approach
is to combined ideas from~\cite{xu16sharing}, which explores composition of
barrier certificates for continuous time systems, with theorems from
Chapter~\ref{chap:exp-smpl}, which explore barrier certificates for
sampled-data systems. This might address a particular shortcoming in
Chapter~\ref{chap:emsoft16}: the composition proof rules did not leverage
any of the differential invariants/barrier certificates used to verify
subcomponents.

\paragraph{Vertical composition}
In Section~\ref{sec:exp-smpl-overview}, we described how the real
implementation of the Ardupilot controller does not directly set
acceleration but instead outputs desired velocity commands to an inner loop
velocity controller. To simplify reasoning the outer loop controller was
designed under the approximation that the inner loop controller
instantaneously achieves any desired velocity command. The informal
justification for this approximation is that, as long as the velocity
commands do not change too quickly, the velocity dynamics are sufficiently
fast relative to position dynamics that the approximation error will not
cause significant violations of position constraints. It would be valuable
to formalize this argument: under what conditions can the position
constraint-enforcing outer loop controller be vertically composed with the
inner loop velocity controller, and how much constraint violation does this
cause?

Answering this question will require determining what properties the inner
loop velocity controller must satisfy. The informal argument is based on
the velocity dynamics being sufficiently fast -- this suggests that the
inner loop controller must satisfy a fast convergence property, such as
local exponential stability. It would be valuable to verify such a property
in the same formal framework as this dissertation, suggesting that the Coq
framework needs to be extended with rules for reasoning about stability
properties from control theory, particularly for sampled-data systems.
Another promising approach is to explore the use of reference
governors~\cite{bemporad98reference}, a concept from control theory in an
inner loop controller is augmented with a module (the reference governor)
that modifies commands (references) to the controller in order to guarantee
desired state constraints.

\paragraph{Verified implementations}
All of the systems verified in this dissertation have contained high-level
models of controllers rather than executable code. Our own experience has
demonstrated that bugs can occur at all levels of the development process,
not just in the high-level design. Based on the ideas in
Chapter~\ref{chap:exp-smpl}, we implemented a geofence module for Ardupilot
that prevents a vehicle from leaving a specified safe area, regardless of
what the pilot does. This important safety feature required extending the
one-spatial dimension constraints from Chapter~\ref{chap:exp-smpl} to an
arbitrary 2D polygon and to a circular fence. At the core of this
implementation is a constraint solver that chooses a safe velocity as close
as possible to the pilot's command while satisfying the constraints of each
edge of the fence (or circle). Our original C++ implementation passed the
review process of the Ardupilot developers, was merged into the
development, and run in the current public release of
Ardupilot. Unfortunately, we later discovered that the constraint solving
implementation is incorrect.\footnote{The bug is documented in the
  following github issues:
  \url{https://github.com/ArduPilot/ardupilot/issues/4429} and
  \url{https://github.com/ArduPilot/ardupilot/issues/4807}.}

This experience suggests that it is valuable to have end-to-end formal
verification results, from the high level design all the way to the low
level implementation. Coq is well suited for this task, as it is expressive
enough for both the high level models (as demonstrated in this
dissertation) as well as executable code
verification~\cite{leroy2009compcert,appel2014vst}, even including floating
point arithmetic~\cite{flocq11}. Such an end-to-end verification result
would be challenging for a standalone hybrid system verification tool as it
would need to interface with another tool for formalizing and reasoning
about code. We expect that a shallow encoding of a hybrid system logic in
Coq will facilitate this task, as it will ease the process of integrating
with other Coq developments for reasoning about low level code, such
as~\cite{appel2014vst,flocq11}.

\paragraph{Probability}
All controllers in this dissertation require an estimate of the state of
the system. Since sensors are not perfect, these estimates are inherently
probabilistic. In Chapter~\ref{chap:memo15}, we presented an approach to
reason about systems in the presence of sensor error when there is a
concrete bound on that error. However, this can be overly pessimistic in
certain cases and inaccurate in others. The core sensor fusion algorithms
used to estimate a system's state based on sensor readings (e.g. Kalman
filters) are based on probability~\cite{Simon06estimation}. It would be
valuable to formalize these approaches and then reason about state
constraint enforcing controllers in the presence of probability. Again, the
expressiveness of Coq provides a promising framework for this direction.

\paragraph{More examples and other domains}
We evaluated the sampled-data proof rules in this dissertation on various
instantiations of a single important application: the double
integrator. While this is a benchmark application, it would be valuable to
evaluate the proof rules on other sampled-data systems. A promising
starting point would be the large collection of sampled-data systems that
have been verified in the KeYmaera and KeYmaera X proof assistants,
e.g.~\cite{platzercruise11,platzerrobots13}. This will help build more
evidence and understanding of when domain-specific proof rules simplify
reasoning and reduce the proof burden for sampled-data systems.

In addition, while sampled-data can be used to model many real world
systems, there are many systems that fit outside the domain. One
interesting example is the domain of event-triggered
systems.\footnote{Sampled-data systems are sometimes called time-triggered
  systems.} There is potential for domain-specific proof rules to simplify
reasoning for this and other classes of systems. Developing these rules
could be an important step towards making formal verification a practical
technique for real cyber-physical systems.


%\appendix

% Stuff at the end of the dissertation goes in the back matter
\backmatter
\bibliographystyle{plain} % Or whatever style you want like plainnat
\bibliography{bib,platzer,intro}

\end{document}
