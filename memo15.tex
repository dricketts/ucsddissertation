This chapter presents two proof rules for reasoning about sampled-data
systems. The first rule is an induction rule specialized to sampled-data
systems. This rule decomposes an inductive safety proof into a proof about
the initial state, a proof about the discrete transition, and a proof about
the continuous transition. For each of the transitions, the proof rule
introduces timing constraints that are guaranteed by the periodic
sampled-data model. Such a rule automatically manages the aspects of an
inductive safety proof that are common to all periodic sampled-data
systems, allowing the user to focus on the application specific aspects of
the proof. Without such a proof rule, this tedious decomposition would have
to be done manually.

The second rule provides a simple mechanism for composing systems with
non-cyclic dependencies. This allows one component to assume that invariant
of another when proving its own invariant. We demonstrate how this can be
used to reason about systems in the presence of both sensor error and delay
by chaining instances of this rule together. Chapter~\ref{chap:emsoft16}
presents an alternate approach that removes the non-cyclic restriction.

We introduce the proof rules along side a running example: a controller
that prevents a quadcopter from violating some maximum velocity.  We
additionally used the rules to verify a controller that prevents a
quadcopter from violating some maximum height, and used our composition
rule to verify them both in the presence of sensor error and delay.

In an effort to ensure that our examples are realistic, we implemented them
on a actual quadcopter and performed flight tests. Doing so forced us to
confront an important issue. Since our controllers are components of a much
larger system containing numerous controllers, how do we architect the
autopilot to incorporate our controllers? Throughout the paper, we describe
our solution and discuss engineering tradeoffs. This solution has not been
formally verified. That is, we have not formally modeled the entire
autopilot and verified the properties of our controllers in that
context. We view this as an important direction for future work. We also
discuss discrepancies between our model and reality as well as lessons
learned from doing foundational verification of sampled-data systems.

